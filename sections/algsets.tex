\section{Algebraic sets}
In the following, we will consider an algebraically closed field \(k = \cl{k}.\)
\begin{df}
  An \emph{algebraic set} in \(k^n\) is a subset of the form
  \[V(E) = \{\alpha \in k^n \suchthat \forall f \in E \enspace f(\alpha) = 0\},\]
  where \(E \subseteq S = k[x_1, \dotsc, x_n]\) is some subset. \(V(E)\) is then also called the \emph{vanishing locus} of \(E.\)
\end{df}

\begin{prop}
  \label{algebraic sets ideals are enough}
  \(V(E) = V(I),\) where \(I = (E)\) is the ideal generated by \(E.\)
\end{prop}
\begin{proof}
   Indeed: if \(f, g \in k[x_1, \dotsc, x_n]\) and \(f(\alpha) = 0,\) then also \((f \cdot g)(\alpha) = f(\alpha) \cdot g(\alpha) = 0 \cdot g(\alpha) = 0.\)
\end{proof}

\begin{lemma}
  \label{zariski algebraic sets}
  The sets
  \(\{V(E) \subseteq k^n \suchthat E \subseteq k[x_1, \dotsc, x_n]\}\)
  are closed subsets of a topology.
\end{lemma}
\begin{proof}
  One needs to check that:
  \begin{enumerate}
  \item \(\exists E \subseteq k[x_1, \dotsc, x_n] \enspace V(E) = k^n,\)
  \item \(\exists E \subseteq k[x_1, \dotsc, x_n] \enspace V(E) = \emptyset,\)
  \item \(\forall I \enspace \forall (E_i)_{i \in I} \subseteq (k[x_1, \dotsc, x_n])^I \enspace \exists E \subseteq k[x_1, \dotsc, x_n] \enspace V(\bigcap_{i \in I} E_i) = V(E),\)
  \item \(\forall E_1, E_2 \subseteq k[x_1, \dotsc, x_n] \enspace \exists E \subseteq k[x_1, \dotsc, x_n] \enspace V(E_1) \cup V(E_2) = V(E).\)
    \end{enumerate}
  One inevitably finds that:
  \begin{enumerate}
  \item \(E = \emptyset\) does the job since any point satisfies an empty set of conditions,
  \item \(E = \{(1\}\) does the job,
  \item if one puts \(E = \bigsetsum_{i \in I} E_i,\)
    then, for any \(\alpha \in k^n:\)
    \begin{align*}
      \alpha \in V(E) & \iff \forall f \in \bigsetsum_{i \in I} \enspace f(\alpha = 0) \\
                      & \iff \forall i \in I \enspace \forall f \in E_i \enspace f(\alpha) = 0 \\
                      & \iff \forall i \in I \alpha \in V(E_i) \\
                      & \iff \alpha \in \bigsetintersection_{i \in I} V(E_i),
    \end{align*}
  \item if we put \(E = E_1 \cdot E_2 = \{f_1 \cdot f_2 \suchthat f_1 \in E_1, f_2 \in E_2\},\) then for any \(\alpha \in k^n\) the following holds:
    \begin{align*}
      \alpha \in V(E) & \iff \forall f \in E_1 \cdot E_2 \enspace f(\alpha = 0) \\
                      & \iff \forall f_1 \in E_1, f_2 \in E_2 \enspace (f_1 \cdot f_2)(\alpha) = f_1(\alpha) \cdot f_2(\alpha) = 0 \\
                      & \iff \forall f_1 \in E_1, f_2 \in E_2 \enspace f_1(\alpha) = 0 \enspace \lor f_2(\alpha) = 0 \\
                      & \iff \alpha \notin V(E_1) \implies \alpha \in V(E_2) \enspace \land \enspace \alpha \notin V(E_2) \implies \alpha \in V(E_1) \\
                      & \iff \alpha \in V(E_1) \enspace \lor \alpha \in V(E_2) \\
                      & \alpha \in V(E_1) \cup \alpha \in V(E_2)
    \end{align*}
  \end{enumerate}
\end{proof}


\begin{df}
  The topology defined by \cref{zariski algebraic sets} is called the \emph{Zariski topology} on \(k^n.\)
\end{df}

\begin{df}
  For \(Z \subseteq k^n\) one defines
  \(I(Z) = \{f \in k[x_1, \dotsc, x_n] \suchthat f(Z) = \{0\}\}.\)
\end{df}

\begin{example}
  Suppose \(\alpha = (\alpha_1, \dotsc, \alpha_n) \in k^n\) and \(Z = \{\alpha\}.\) Then \[I(\{\alpha\}) = \{f \in k[x_1, \dotsc, x_n] \suchthat f(\alpha) = 0\} = (x_1 - \alpha_1, \dotsc, x_n - \alpha_n).\]
\end{example}

\begin{lemma}
  If \(Z \subseteq k^n\) is any subset, then
  \[I(Z) = \bigcap_{\alpha \in Z} I(\{\alpha\}) = \bigcap_{\alpha \in Z} (x_1 - \alpha_1, \dotsc, x_n - \alpha_n).\]
\end{lemma}
\begin{proof}
  Straightforward.
\end{proof}

\begin{df}
  An ideal \(I \subseteq R\) is called radical if \(\rad{I} = I.\)
\end{df}

If \(J = \rad{I}\), then \(\rad{J} = J\), and so \(J\) is radical.

\begin{lemma}
  For any subset \(Z \subseteq k^n\), \(I(Z) \subseteq k[x_1, \dotsc, x_n]\) is a radical ideal.
\end{lemma}
\begin{proof}
  If \(f^n \in I(Z),\) then \(f^n(Z)  = \{0\}\). This happens if and only if \(f(Z) = \{0\}\), hence exactly when \(f \in I(Z).\)
\end{proof}


\begin{theorem}[Nullstellensatz, algebraic set version]
  \label{algebraic set nullstellensatz}
  For any ideal \(J \subseteq k[x_1, \dotsc, x_n]\) and subset \(Z \subseteq k^n\), one has
  \[I(V(J)) = \rad{J} \text{ and } V(I(Z)) = \cl{Z}.\]
\end{theorem}
\begin{proof}
  Pick any \(\alpha \in k^n.\) Then
  \begin{align*}
    \alpha \in V(J) & \iff \forall f \in J \enspace f(\alpha) = 0 \\
                    & \iff \forall f \in J \enspace f \in I(\alpha) \\
                    & \iff J \subseteq I(\alpha) = (x_1 - \alpha_1, \dotsc, x_n - \alpha_n).
  \end{align*}
  Then
  \[I(V(J)) = \bigcap_{\alpha \in V(J)} I(\alpha) = \bigcap_{J \subseteq I(\alpha)} I(\alpha) = \bigcap_{J \subseteq \mathfrak m} \mathfrak m = \rad{J},\]
  in which the second to last equality is implied by the weak Nullstellensatz (\cref{weak Nullstellensatz}) and the last - by the strong Nullstellensatz (\cref{Nullstellensatz}).

  For the other equality, we consider
  \[\cl{Z} = \bigcap_{Z \subseteq V} V = \bigcap_{Z \subseteq V(J)} V(J) = \bigcap_{J \subseteq \bigcap_{\alpha \in Z} I(\alpha)} V(J) = \bigcap_{J \in I(Z)} V(J) = V(I(Z)).\]
\end{proof}

\begin{corollary}
  \label{closed in bijection with radical}
  The functions \(I(-)\) and \(V(-)\) give a bijective correspondence
  \[\{Z \subseteq k^n, \text{$Z$ closed}\} \cong \{J \subseteq S \suchthat \rad{J} = J\}.\]
\end{corollary}

\begin{note}
  If \(k\) is not algebraically closed, then the weak Nullstellensatz and the relaed results may well fail. For example, one can produce a polynomial \(f \in k[x_1, \dotsc, x_n]\) such that
  \(V(f) = \{(0, \dotsc, 0)\} \subseteq k^n).\)
\end{note}

\begin{df}
  An algebraic set \(V\) is irreducible if it is irreducible in the Zariski topology; that is, if for any pair of algebraic sets \(V_1, V_2\), the equality \(V = V_1 \cup V_2\) implies that either \(V_1 = V\) or \(V_2 = V.\)
\end{df}

\begin{prop}
  \label{prime irreducible}
  The bijection of \cref{closed in bijection with radical} restricts to
  \[\{Z \subseteq k^n \suchthat \text{$Z$ irreducible closed}\} \cong \{\mathfrak p \subseteq k[x_1, \dotsc, x_n] \suchthat \text{$\mathfrak p $ prime}\}.\]
\end{prop}
\begin{proof}
  Take a radical ideal \(J = \rad{J}.\) We state two claims:
  \begin{enumerate}
  \item If \(J\) is prime, then \(V(J)\) is irreducible.
  \item If \(J\) is not prime, then \(V(J)\) is reducible.
  \end{enumerate}
  For the first, one, write \(V(J) = V_1 \cup V_2.\) By \cref{algebraic set nullstellensatz},
  \(J = I(V(J)) = I(V_1 \cup V_2) = I(V_1) \cap I(V_2).\)
  Hence,
  \(I(V_1) \cdot I(V_2) \subseteq J.\)
  Because \(J\) is prime, it follows that either \(I(V_1) \subseteq J\) or \(I(V_2) \subseteq J\) (again by \cref{algebraic set nullstellensatz}).
  In any case, it follows that \(V_1 = V(J)\) or \(V_2 = V(J)\), and so \(V(J)\) is irreducible since \(V_1, V_2\) were arbitrary.
  % TODO finish argument

  For the second claim, pick \(a, b \notin J\) such that \(ab \in J\). We let
  \(V_1 = V((a) + J),\)
  \(V_2 = V((b) + J).\)
  Now, we write
  \[V_1 \cup V_2 = V((a) + J) \cup V((b) + J) = V(J) \subseteq V(((a) + J) \cdot ((b) + J)).\]
  In particular,
  \(V(J) \subseteq V_1 \cup V_2).\)
  Suppose that
  \(V_1 = V(J).\) Then \(I(V_1) = I(V(J)) = J.\)
  But \(a \in I(V_1)\) and \(a \notin J\).
  Analogously, \(V_2 = V(J)\) leads to contradiction.
\end{proof}

% In a way, \cref{prime irreducible} allows us to think of
% forgot the intuition

\begin{example}
  Let \(n = 1,\) \(S = k[x].\) Then the irreducible subseteq of \(k^1\) are:
  \begin{itemize}
  \item \(k^1 = \cl{V(0)}),\)
  \item \(V(x-\lambda), \enspace \lambda \in k.\)
  \end{itemize}
\end{example}

\begin{example}
  Let \(n=2,\) \(S = k[x_1, x_2].\)
  This is a unique factorization domain. The prime ideals are:
  \begin{itemize}
  \item \((0)\),
  \item \((f)\) such that \(f\) is irreducible,
  \item \((x_1 - \alpha_1, x_2 - \alpha_2)\) such that \((\alpha_1, \alpha_2) \in k^2\), that is, the maximal ideals.
  \end{itemize}
  Recall that \(\dim(S) = 2.\) One sees easily that the points above make up all chains of prime ideals; indeed, if \(\mathfrak p\) is a prime ideal that is not zero and not maximal, then one may pick a nonzero element \(f \in \mathfrak p\) along with its decomposition into primes \(f = f_1 \cdot \dotsm \cdot f_r\). Then one of the \(f_i\) is irreducible, \((f_i) \subseteq \mathfrak p\), and if \(\mathfrak p \neq (f_i),\) this would expose a chain of length \(3\), which is absurd.

  As we had remarked previously, considerations like this make precise the intuition that the dimension \(k\) components of an algebraic set are the \(k\)-dimensional surfaces inside the algebraic set.
\end{example}

\begin{lemma}
  \(k^n\) equipped with the Zariski topoloy is a Noetherian topological space.
\end{lemma}
\begin{proof}
  Consider a descending sequence \(V_1 \supseteq V_2 \supseteq \dotso\) of closed sets; this leads to an ascending sequence of ideals
  \[I(V_1) \subseteq I(V_2) \subseteq \dotso \subseteq k[x_1, \dotsc, x_n].\]
  This stabilizes, because \(k[x_1, \dotsc, x_n]\) is Noetherian. Suppose is stabilizes at \(k\); then
  \(\forall l \geq k V(I(V_l) = V(I(V_k))).\)
  But \(V(I(V_i)) = V_i\) by the Nullstellensatz (\cref{algebraic set nullstellensatz}), since \(V_i\) are closed,
  % TODO correct the reference
  and so the claim is proved.
\end{proof}

\begin{corollary}
  For any ideal \(J \subseteq k[x_1, \dotsc, x_n],\) there are prime ideals
  \(\mathfrak p_1, \dotsc, \mathfrak p_r \subseteq k[x_1, \dotsc, x_n]\)
  such that
  \(\rad{J} = \mathfrak p_1 \cap \dotsb \cap \mathfrak p_r.\)
  In other words, any radical is an intersection of finitely many prime ideals.
\end{corollary}
\begin{proof}
  \(V(J) = V(\rad{J}) = V_1 \cup \dotsb \cup V_r\) with \(V_i\) irreducible algebraic sets. Then
  \[\rad{J} = I(V(J)) = I(V_1 \cup \dotsb \cup V_r) = I(V_1) \cap \dotsb \cap I(V_r),\]
  and the \(I(V_i) = \mathfrak p_i\) are in fact prime ideals, because \(V_i\) is irreducible.
  % TODO fix reference
\end{proof}






%%% Local Variables:
%%% mode: latex
%%% TeX-master: "../commalg"
%%% End:
