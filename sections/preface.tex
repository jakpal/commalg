\documentclass[10pt,a4paper, openany, bibliography=totoc, egregdoesnotlikesansseriftitles, oneside]{scrbook}
\usepackage{amsfonts}
\usepackage{amsmath}
\usepackage{amsthm}
\usepackage{amssymb}
\usepackage{bbm}
\usepackage{geometry}
\usepackage{thmtools}
\usepackage{verbatim}
\usepackage{enumerate}
\usepackage[shortlabels]{enumitem}
\usepackage{microtype}
\usepackage{tikz-cd}
\usepackage{csquotes}
\usepackage{booktabs}
\usepackage{centernot}
\usepackage{mathtools}
\usepackage{xparse}
\usepackage{graphicx}
\usepackage{yfonts}

% If using AucTeX, run C-c biber whenever bibliography is changed; due to a bug, biber is not recognized as a backend.
\usepackage[backend=biber,style=alphabetic]{biblatex}
\bibliography{references.bib}

% keep these at the end, and definitions of theorems and their counters after that
\usepackage{hyperref}
\usepackage[capitalise, noabbrev]{cleveref}


% stops line-breaking inline mathematics
% \binoppenalty = \maxdimen
% \relpenalty = \maxdimen
% also requires discipline in typing

\PassOptionsToPackage{hyphens}{url}\usepackage{hyperref}

\newcounter{alltheorems}
\numberwithin{alltheorems}{chapter}

\theoremstyle{plain}
% daje brzydkie pochylone litery

\theoremstyle{definition}
\newtheorem{exercise}[alltheorems]{Exercise}
\newtheorem{df}[alltheorems]{Definition}
\newtheorem{example}[alltheorems]{Example}
\newtheorem{construction}[alltheorems]{Construction}
\newtheorem{lemma}[alltheorems]{Lemma}
\newtheorem{theorem}[alltheorems]{Theorem}
\newtheorem{corollary}[alltheorems]{Corollary}
\newtheorem{prop}[alltheorems]{Proposition}

\newtheorem*{answer}{Answer}

\theoremstyle{remark}
\newtheorem{note}[alltheorems]{Note}
\newtheorem{question}[alltheorems]{Question}
\newtheorem{intuition}[alltheorems]{Intuition}

\declaretheoremstyle[
spaceabove=6pt,
spacebelow=6pt
headfont=\normalfont\bfseries,
headpunct={.} ,
bodyfont=\normalfont,
]{exobreak}

\declaretheorem[style=exobreak, name=Exercise]{ex}

% \newcommand{\ohne}{\backslash}
% use \setminus instead
\newcommand{\naturals}{\mathbb N}
\newcommand{\integer}{\mathbb Z}
\newcommand{\rational}{\mathbb Q}
\newcommand{\real}{\mathbb R}
\newcommand{\complex}{\mathbb C}
\newcommand{\projective}{\mathbb P}
\newcommand{\var}{\varepsilon}
\newcommand\mapsfrom{\mathrel{\reflectbox{\ensuremath{\mapsto}}}}
\newcommand{\outerp}{\Lambda}
\newcommand{\smashp}{\wedge}
\newcommand{\letequal}{\coloneqq}
\newcommand{\equallet}{\eqqcolon}
\newcommand{\spec}{\text{Spec}}
\newcommand{\diag}{\text{diag}}
\newcommand{\id}{\text{id}}
\newcommand{\tensor}{\otimes}
\newcommand{\existsunique}{\exists \text{!}}
\newcommand{\set}{\text{Set}}
\newcommand{\module}[1]{#1_\text{mod}}
\newcommand{\disjointunion}{\bigsqcup}
\newcommand{\dif}{\text{d}}
\newcommand{\gal}{\text{Gal}}
\newcommand{\rad}[1]{\sqrt{#1}}
\newcommand{\cl}[1]{\overline{#1}}
\newcommand{\trdeg}{\mathrm{trdeg}}
\newcommand{\fractions}{\mathrm{Frac}}

% arrows
\newcommand{\xto}{\xrightarrow}
\newcommand{\xlongto}[1]{\xlongrightarrow{\;#1\;}}
\newcommand{\into}{\hookrightarrow}
\newcommand{\onto}{\twoheadrightarrow}
\newcommand{\toin}{\hookleftarrow}
\newcommand{\xra}[1]{\xrightarrow{#1}}
\newcommand{\xla}[1]{\xleftarrow{#1}}
\newcommand{\ot}{\leftarrow}

\newcommand{\suchthat}{\mathrel{|}}

\DeclareMathOperator*{\aut}{Aut}
\DeclareMathOperator*{\im}{im}
\DeclareMathOperator*{\rank}{rk}
\DeclareMathOperator*{\colim}{colim}
\DeclareMathOperator*{\lin}{lin}
\DeclareMathOperator*{\en}{End}
\DeclareMathOperator{\bilin}{Bilin}
\DeclareMathOperator{\der}{Der}
\DeclareMathOperator{\nil}{Nil}
\DeclareMathOperator{\setsum}{\cup}
\DeclareMathOperator*{\bigsetsum}{\bigcup}
\DeclareMathOperator{\setintersection}{\cap}
\DeclareMathOperator*{\bigsetintersection}{\bigcap}

\DeclarePairedDelimiter{\abs}{\lvert}{\rvert}

\NewDocumentCommand{\restrict}{m m O{}}{#1|_{#2}^{#3}}

%%% Local Variables:
%%% mode: latex
%%% TeX-master: "../commalg"
%%% End:
