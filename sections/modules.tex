\section{Modules}

\begin{df}
  Let $A$ be a ring. An $A$-module $M$ is an abelian group $M$ together with a ring homomorphism $A \to \en_{\integer}(M)$.

Equivalently, $M$ is an abelian group together with a map
\[ A \times M \to M, \quad (a, m) \mapsto am, \]
such that:
\begin{enumerate}
\item $\forall a \in A, \enspace m_1, m_2 \in M \enspace a(m_1+m_2) = am_1 + am_2$
\item $(a_1 + a_2)m = a_1 m + a_2 m$
\item $1 \cdot m = m$
\item $a_1 (a_2 m) = (a_1 a_2) m$
\end{enumerate}
\end{df}

\begin{df}
  A homomorphism of $A$-modules $\phi \colon M \to N$ is a homomorphism of abelian groups such that
  \[ \forall a \in A \enspace \forall m \in M \enspace a \phi(m) = \phi(am). \]
\end{df}

\begin{example}
  For $A$ a ring, $A$ is an $A$-module; in fact, any ideal $I \subseteq A$ is an $A$-module.
\end{example}

\begin{example}
  For any $A$-algebra $B$, $B$ is an $A$-module under the action \(a.b \coloneqq f(a) \cdot b,\) where \(f\) is the structural homomorphism. In particular, $A/{I}$ is an $A$-module for any ideal $I \subseteq A$.
\end{example}

\begin{df}
  If $M$ is an $A$-module and $m_1, \ldots, m_k \in M$ its elements, then the submodule generated by $m_1, \ldots, m_k$ is
  \[ Am_1 + Am_2 + \ldots + Am_k = \{ \sum_{i=1}^k a_i m_i \suchthat a_i \in A \}.\]
\end{df}

\begin{prop}
  The set
  \[ \hom_A(M, N) = \{ \phi: M \to N \text{ an $A$-module homomorphism} \} \]
  is an $A$-module under
  \[ (a.\phi)(m) \coloneqq \phi(am) = a \phi(m).\]
\end{prop}

\begin{df}
  An $A$-module is finitely generated if
  $\exists k \in \naturals, \enspace m_1, \ldots, m_k \in M$
  such that
  \[M = A_1 + \ldots + A_{m_k}.\]
\end{df}

\begin{df}
  An $A$-module is free if it is isomorphic to an $A$-module of the form
  \[ \bigoplus_{i \in I} A.\]
\end{df}

\begin{example}
  All $k$-vector spaces are free $k$-modules. Not all abelian groups are free.
\end{example}

\begin{lemma}
  An $A$-module $M$ is finitely generated if and only if there exists a surjective homomorphism
  \[ \phi: \bigoplus_{i=1}^k A \to M.\]
\end{lemma}

%%% Local Variables:
%%% mode: latex
%%% TeX-master: "../commalg"
%%% End:
