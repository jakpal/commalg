\section{First definitions}

\begin{df}
  Let $A$ be a ring. Then an \textit{$A$-algebra} is a ring $B$ together with a fixed homomorphism
  \[A \to B.\]
  This homomorphism is called the \textit{structural homomorphism}.
\end{df}

\begin{df}
  Let $A$ be a ring. A \textit{homomorphism of $A$-algebras} is a ring homomorphism commuting with the structural maps, that is, if
  \[ \phi: A \to B, \quad \psi: A \to C\]
  are $A$-algebras, then a ring homomorphism
  $f:B \to C$
  is an algebra homomorphism if
  \[\psi = f \circ \phi.\]
\end{df}

\begin{example}
  Let $f: \complex \to \complex$ be given by $z \mapsto \bar z$ (the complex conjugation). Then:
\begin{itemize}
\item \(f\) is a ring homomorphism,
\item \(f\) is a real algebra homomorphism,
\item \(f\) is not a complex algebra homomorphism.
\end{itemize}
\end{example}

\begin{example}
  The polynomial ring $\complex [x]$ becomes a $\complex$-algebra under the inclusion map onto the zero degree component.

  In this case, if $I$ is an ideal, then 
  the quotient map
  \[ \complex[x] \to \complex[x]/{I}\]
  is a complex algebra homomorphism.
\end{example}

\begin{df}
\mbox{}
\begin{enumerate}
\item A ring $A$ is a domain if
  \[\forall a,b \in A \enspace ( ab = 0 \implies a = 0 \lor b = 0. )\]
\item An ideal $I$ in a ring $B$ is prime if $B/{I}$ is a domain.
\item A ring $A$ is a field if
  \[\forall 0 \neq a \in A \enspace \exists b \in A \enspace ab = 1,\] that is, every nonzero element has a multiplicative inverse.
\item An ideal $I$ in a ring $B$ is maximal if $B/{I}$ is a field.
\end{enumerate}
\end{df}


%%% Local Variables:
%%% mode: latex
%%% TeX-master: "../commalg"
%%% End:
