\section{Krull dimension}

\begin{df}
  Let $A$ be a ring. A \textit{chain of prime ideals} in $A$ is a sequence of proper inclusions
  \[ \mathfrak p_0 \subseteq \mathfrak p_1 \subseteq \dotso \subseteq \mathfrak p_s\]
  of prime ideals.

  The \textit{length} of such a chain is the number of those inclusions, that is, a number smaller by one than the number of ideals in the chain.
\end{df}

\begin{df}
  The \textit{Krull dimension} of $A$ is
  \[ \dim A = \sup \{ \text{ length of a chain of prime ideals in $A$} \}.\]
\end{df}

\begin{example}
  \mbox{}
  \begin{enumerate}
  \item If $k$ is a field, then $\dim k = 0$.
  \item If $A$ is a principal ideal domain, then $\dim A = 1$, since nonzero prime ideals are maximal.
  \end{enumerate}
\end{example}

Clearly, \(\dim(k[x_1, \dotsc, x_n]) \geq n,\) as a chain of length $n$ is given as
\[ (0) \subseteq (x_1) \subseteq (x_1, x_2) \subseteq \dotso \subseteq (x_1, \dotsc, x_n).\]
However, the opposite inequality if more difficult to prove since $k[x_1, \dotsc, x_n]$ has many more prime ideals than those. Nevertheless, we aim to tackle this problem in the immediate future.

\begin{example}
  Let $f = x_n^3 + x_n^2 x_1$. One can take $x_i' = x_i$ for $i=1, 2, \dotsc, n-1$. Then
  \[ x_n^3 + x_1' x_n^2 - f = 0\]
  and the extension $k[x_1', \dotsc, x_{n-1}', x_n \cdot f] \subseteq k[x_1, \dotsc, x_n]$ is finite.
\end{example}

This approach can be extended in a way allowing us to prove \cref{poly dimension}.

\begin{lemma}[Nagata's trick]
  \label{nagata}
  Let $f \in S = k[x_1, \dotsc, x_n]$, $f \notin k$. Then there exist \[x_1', \dotsc, x_{n-1}' \in S\] such that
  \[k[x_1', x_2', \dotsc, x_{n-1}', f] \subseteq k[x_1, \dotsc, x_n]\]
  is a finite extension.
\end{lemma}
\begin{proof}
  Fix $e > \deg(f)$ and assume without loss of generality that $x_n$ appears in $f$. Consider a $k$-algebra homomorphism
  \[ \phi \colon S \to S \quad \text{via} \quad x_i \mapsto x_i - x_n^{e^{n+1-i}}, \enspace x_n \mapsto x_n.\]
  We claim that $\phi$ is an isomorphism with inverse
  \[ \phi^{-1} \colon S \to S \quad \text{via} \quad x_i \mapsto x_i + x_n^{e^{n+1-i}} \enspace x_n \to x_n.\]
  Moreover, $\phi(f) = \pm x_n^D + \text{ monomials of a smaller total degree}$.

  Indeed, one sees that distinct monomials are mapped by $\phi$ to polynomials with different top degree of $x_n$. More specifically, if $m = x_1^{a_1} \cdot \dotsb \cdot x_n^{a_n}$ is such that $a_n > 0$ and $m$ is lexicographically the greatest possible of all monomials in $f$, then
  \[ D = \sum_{i=0}^n a_i e^{n-i}.\]
  The process is akin to coding the variables in base $e$, and our claim is that decoding remains possible.

  One takes $x_i' = \phi^{-1}(x_i)$. Then all $x_i$ are integral over $k[x_1', \dotsc, x_{n-1}', \phi(f)$. One may look at the diagram
  \[
  \begin{tikzcd}[column sep = huge]
    k[x_1, \dotsc, x_{n-1}, \phi(f)] \arrow{r}{\text{integral}} \arrow{d}{\phi^{-1}}
    & k[x_1, \dotsc, x_n] \arrow{d}{\phi^{-1}} \\
    k[x_1', \dotsc, x_{n-1}', f] \arrow{r}
    & k[x_1, \dotsc, x_n]
  \end{tikzcd}
  \]
  The horizontal inclusions are integral, and because $B$ is finitely generated over $A$, the extension must be finite by \cref{lem-integral-fingen-fin}.
\end{proof}

\begin{prop}
  \label{going-up}
  If $A \into B$ is integral, then $\dim A = \dim B$.
\end{prop}
\begin{proof}
  First, we show $\dim A \leq \dim B$.
  Take a chain of prime ideals in $A$
  \[ \mathfrak p_0 \subseteq \mathfrak p_1 \subseteq \dotso \subseteq \mathfrak p_s.\]
  We will lift it to a chain in $B$. We shall proceed by induction:
  \begin{itemize}
  \item The base case follows from surjectivity of $\spec(B) \to \spec(A)$.
  \item For the induction step, consider a partial lift
    \[ \mathfrak q_0 \subseteq \dotso \subseteq \mathfrak q_k.\]
    We wish to find $\mathfrak q_{k+1}$ such that $\mathfrak q_{k+1} \cap A = \mathfrak p_{k+1}$.
    Consider $A/{\mathfrak p_k} \to B/{\mathfrak q_k}$; this is injective, and so it is surjective by
    % TODO
    .
    Then one picks a preimage of $p_{k+1}$ in $B/{\mathfrak q_k}$.
  \end{itemize}
  For $\dim A \geq \dim B$, take
  \[ \mathfrak q_0 \subseteq \dotso \subseteq \mathfrak q_s\]
  a chain in $B$; take $p_i = \mathfrak q_i \cap A$. Then
  \[ \mathfrak p_0 \subseteq \dotso \subseteq \mathfrak p_s\]
  is a chain in $A$, as if $\mathfrak p_i = \mathfrak p_{i+1}$, $\mathfrak q_i \cap A = \mathfrak q_{i+1} \cap A$ and then by incomparability
  % TODO give ref
  $\mathfrak q_i = \mathfrak q_{i+1}$.
\end{proof}

\begin{theorem}
  \label{poly dimension}
  If $k$ is a field, then
  \[\dim (k[x_1, \dotsc, x_n]) = n.\]
\end{theorem}
\begin{proof}
  Proceed by induction on $n$. Base $n=0$ is fine.

  Now assume that we have shown the claim for $n-1$ want to prove for $n$. Suppose that this is false, and so we get a chain of length $n+1$. For $i \geq 1$, pick \(f \in \mathfrak p_i.\)
  Use Nagata's trick (\cref{nagata}) to get a finite extension
  \[ A = k[x_1', \dotsc, x_{n-1}' \cdot f] \subseteq k[x_1, \dotsc, x_n].\]
  Take a preimage of a chain on the right side in the left side.
  Consider a quotient map
  \[ k[y_1, \dotsc y_{n-1}] \to k[x_1', \dotsc, x_n' \cdot f]/{(f)}.\]
  Put $\bar{\mathfrak q}_i = \mathfrak q_i /{f}$.
  If we let $\mathfrak r_i$ to be the preimage of $\bar{\mathfrak q}_i$ in $k[y_1, \dotsc, y_{n-1}]$, then
  \( \mathfrak r_1 \subseteq \mathfrak r_2 \subseteq \dotso \subseteq \mathfrak r_{n+1}\)
  is a chain of length $n$, in contradiction of the inductive assumption.
\end{proof}

\begin{corollary}
  \label{cor poly dimension}
  If
  \(A = k[x_1, \dotsc, x_d] /{I}\)
  with
  \(I \neq 0,\)
  then
  \(\dim(A) < d.\)
\end{corollary}
\begin{proof}
  Suppose
  \(\dim(A) \geq d.\)
  Choose a chain of prime ideals
  \[\mathfrak p_0 \subseteq \mathfrak p_1 \subseteq \dotso \subseteq \mathfrak p_d\]
  in \(A\) and consider their preimages in
  \(k[x_1, \dotsc, x_d],\)
  that is,
  \[\mathfrak q_1 \subseteq \mathfrak q_2 \subseteq \dotso \subseteq \mathfrak q_d.\]
  This is again a chain of prime ideals. Then by considering \(I\) one gets a chain of length
  \(d +1\)
  in \( k [x_1, \dotsc, x_d],\)
  contradicting \cref{poly dimension}.
\end{proof}

\begin{example}
  Take
  \(A = k[x_1, \dotsc, x_d]/{(f)}.\)
  What is the dimension of \(A\)?
  \begin{enumerate}
  \item \(f = 0  \implies \dim(A) = d,\)
  \item \(f \in k \setminus {0} \implies A = 0,\) whose dimension may be taken to be \(-1\) or \(-\infty,\)
  \item If \(f \notin k,\) then by Nagata's coordinate change there exists a finite extension
    \[k[x_1', \dotsc, x_{d-1}', f] \subseteq k[x_1, \dotsc, x_d].\]
    Now, \(k[x_1', \dotsc, x_{d-1}', f]\) is a polynomial ring, since its dimension is \(d\) and thus \cref{cor poly dimension} implies that there are no relations.

    Because of this, there exists a chain
    \[0 \subseteq (f) \subseteq (f, x_1') \subseteq (f, x_1', x_2') \subseteq \dotso \subseteq (f, x_1', x_2', \dotsc, x_{d-1}')\]
    Now, by \cref{going-up} this chain lifts to a chain
    \[0 \subseteq \mathfrak p_0 \subseteq \mathfrak p_1 \subseteq \dotso \subseteq \mathfrak p_{d-1}.\]
  Because \(f \in \mathfrak p_0,\)
  we get a chain in \(k[x_1, \dotsc, x_n]/{(f)},\)
  giving the dimension of
  \(k[x_1, \dotsc, x_d]/{(f)}\)
  as \(d-1.\) 
  \end{enumerate}
\end{example}

\begin{prop}
  If
  \(I \subseteq k[x_1, \dotsc, x_d]\)
  and \(x_d \in I,\)
  then
  \[k[x_1, \dotsc, x_d]/{I} \cong k[x_1, \dotsc, x_d]/{I \cap k[x_1, \dotsc, x_{d-1}]}.\]
\end{prop}



%%% Local Variables:
%%% mode: latex
%%% TeX-master: "../commalg"
%%% End:
