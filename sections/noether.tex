\section{Noetherian modules}
\begin{prop}
  \label{noetherian-prop}
  Let $M \in \mod{A}$. Then the following conditions are equivalent:
  \begin{enumerate}
  \item every submodule of $M$ is finitely generated,
  \item every sequence of submodules
    \[ M_1 \subseteq M_2 \subseteq \ldots \subseteq M\]
    stabilises, that is,
    \[ \exists n_0 \in \naturals \enspace \forall n \geq n_0 M_n - M_{n_0},\]
  \item every family of submodules of $M$ has a maximal element with respect to inclusion.
  \end{enumerate}
\end{prop}

\begin{proof} (\cref{noetherian-prop}):
  The implication ``$2 \implies 3$'' follows from Kuratowski-Zorn.

  For ``$3 \implies 2$'', take $\{ M_i \}_{i \in \naturals}$ as the family in statement of 3; then the maximal element is also necessarily the one on which the sequence stabilises.

  ``$ 2 \implies 1$'': choose $N \subseteq M$ a submodule. Further, take
  \[ n_1 \in N \setminus \{ 0 \}, \enspace n_2 \in N \setminus A_{n_1}, \enspace n_3 \in N \setminus A_{n_2} \oplus A_{n_3}, \enspace \ldots.\]
  Either one can do this for all $k \in \naturals$ or not. In the latter case, $N$ is necessarily finitely generated. In the former, we get the sequence
  \[ A_{n_1} \subseteq A_{n_1} + A_{n_2} \subseteq A_{n_1} + A_{n_2} + A_{n_3} \subseteq \ldots \]
  that does not stabilize.

  ``$1 \implies 2$'': take a sequence
  \[ M_1 \subseteq M_2 \subseteq \ldots;\]
  we want to stabilise it. To that end, consider
  \[ N = \bigcup_{n \in \naturals} M_n.\]
  Since the sequence is increasing, this is a submodule of $M$.
  
  By $1$, this is finitely generated, and by a finite set of elements from $N$. Then this very set is already contained in one of the $M_k$, and then for $n > k$, $M_n = M_k$.
\end{proof}

\begin{df}
  An $A$-module $M$ is called Noetherian if it satisfies the equivalent conditions of \cref{noetherian-prop}.
\end{df}


\begin{df}
  A ring $A$ is Noetherian if and only if the $A$-module $A$ is Noetherian. 
\end{df}

\begin{prop}
  \label{prime noetherian is noetherian}
  Let $A$ be a ring. The following conditions are equivalent:
  \begin{enumerate}
  \item $A$ is Noetherian,
  \item every ideal of $A$ is finitely generated,
  \item every prime ideal of $A$ is finitely generated.
  \end{enumerate}
\end{prop}

The implication ``$3 \implies 1$'' of \cref{prime noetherian is noetherian} is due to Cohen, 1950.

\begin{prop}
  \label{noetherian-extensions}
  Let $N \subseteq M$ be $A$-modules. Then the following conditions are equivalent:
  \begin{enumerate}
  \item $M$ is Noetherian,
  \item $N$ and $M/{N}$ are Noetherian.
  \end{enumerate}
\end{prop}
\begin{proof}
  We start with ``$1 \implies 2$''. First, we show that $N$ is Noetherian; indeed, if
  \[ N_1 \subseteq N_2 \subseteq \ldots \subseteq N \subseteq M\]
  is a sequence of submodules of $N$, it is also a sequence of submodules of $M$ and the claim follows.

  To show that $M /{N}$ is Noetherian, consider a sequence of submodules
  \[ P_1 \subseteq P_2 \subseteq M /{N}.\]
  If then $\pi \colon M \to M /{N}$ denotes the quotient map, then the sequence
  \[ \pi^{-1}(P_1) \subseteq \pi^{-1}(P_2) \subseteq \ldots \subseteq M\]
  stabilizes, and since $\pi$ is surjective, we have
  \[ \pi(\pi^{-1}(P_n)) = P_n,\]
  ending the proof of ``$1 \implies 2$''.

  For ``$2 \implies 1$'', pick a sequence of submodules
  \[M_1 \subseteq M_2 \subseteq \ldots \subseteq M\]
  and define
  \[ N_k = M_k \cap N, \quad P_k = \pi(M_k).\]
  The sequences defined by the $N_k$ and $P_k$ stabilize, say at common $n_0$.

  We will show that
  \[ \forall n > n_0 \enspace M_n = M_{n_0}.\]
  First, show $M_n \subseteq M_{n_0}$. Take $m \in M_n$, then
  \[ \pi(m) \in P_n = P_{n_0} = M_{n_0} / {\ker \pi \cap M_{n_0}}\]
  Pick $\tilde m \in M_{n_0}$ such that $\pi(m) = \pi(\tilde m)$.
  Then we have
  \[ m - \tilde{m} \in \ker(\pi) \cap M_n = N_n = N_{n_0} \subseteq M_{n_0}.\]
  Since $m-\tilde{m} \in M_{n_0}$, we have that
  in fact
  \[ m = \tilde{m} + m - \tilde{m} \in M_{n_0}\]
  and the claim follows since $m$ was arbitrary in $M_n$.
\end{proof}


Note that this translates to the claim that in the short exact sequence
\[ 0 \to N \to M \to M/{N} \to 0\]
the middle term is Noetherian if and only if the other ones are.

\begin{corollary}
  \label{fin gen noetherian}
  If $A$ is a Noetherian ring, then all finitely generated $A$-modules are Noetherian.
\end{corollary}
\begin{proof}
  First, we show that all finitely generated free modules are Noetherian. This follows easily by induction since we get short exact sequences
  \[ 0 \to A \to A^{\oplus n} \to A^{\oplus (n-1)} \to 0.\]

  If then $K$ is a finitely generated $A$-module, say generated by $k$ elements, we get a short exact sequence
  \[ 0 \to \ker{\alpha} \to A^{\oplus k} \xto{\alpha} K \to 0,\]
  and the claim follows from \cref{noetherian-extensions} again.
\end{proof}


\begin{example}
  Some Noetherian rings include:
  \begin{enumerate}
  \item fields,
  \item principal ideal domains (e.g. $\integer$, $k[x]$, $\integer[x]$).
  \end{enumerate}
\end{example}


\begin{theorem}[Hilbert basis theorem]
  \label{hilbert basis}
  If $A$ is Noetherian, then
  \[ A[x_1, \dotsc, x_n]\]
  is also a Noetherian ring.
\end{theorem}

Note the difference: when considered as a $\complex$-module, $\complex[x]$ is not Noetherian, as it is not finitely generated. However, \cref{hilbert basis} states that as a module over itself, it is Noetherian.

\begin{corollary}
  If $A$ is Noetherian and $B$ is a finitely generated $A$-algebra, then $B$ is also Noetherian.
\end{corollary}
\begin{proof}
  Immediate, since
  \[ B = A[x_1, \dotsc, x_n]/{I}.\]
\end{proof}

One is tempted to ask whether tensor products of Noetherian modules are Noetherian. Indeed, even a slightly stronger result remains true.

\begin{prop}
  \label{tensor fin-gen noetherian}
  Let \(R\) be a ring, \(L\) a finitely generated \(R\)-module and \(N\) a Noetherian \(R\)-module. Then \(L \tensor_R N\) is a Noetherian \(R\)-module.
\end{prop}
\begin{proof}
  That \(L\) is finitely generated means that for some integer \(m > 0,\) there exists a surjection of \(R\)-modules \(R^{\oplus m} \to L;\) this translates to an exact sequence
  \[R^{\oplus m} \to L \to 0.\]
  But the tensor product functor is right-exact (\cref{tensor-right-exact}), and so this leads to an exact sequence
  \[N^{\oplus m} \cong R^{\oplus m} \tensor_R N \to L \tensor_R N \to 0,\]
  the isomorphism on the left side also following from right-exactness.
  Hence, from \cref{noetherian-extensions} it follows that \(L \tensor_R N\) is a Noetherian \(R\)-module, too, as was claimed.
\end{proof}

\begin{corollary}
  If \(R\) is a ring and \(N, M\) are Noetherian \(R\)-modules, \(N \tensor_R M\) is also a Noetherian \(R\)-module.
\end{corollary}
\begin{proof}
  Since a finitely generated \(R\)-module is necessarily Noetherian (\cref{fin gen noetherian}), the claim follows from \cref{tensor fin-gen noetherian}.
\end{proof}

\begin{lemma}
  If the ring $A$ is Noetherian, then so is $A/{I}$ for all ideals $I \subseteq A$.
\end{lemma}

\begin{lemma}
  If $A$ is Noetherian, then for any multiplicative subset $S \subseteq A$, the localization $S^{-1}A$ is also Noetherian.
\end{lemma}
\begin{proof}
  Consider $I \subseteq S^{-1}A$, $J = i^{-1}(I) \subseteq A$.
  Write
  \[ I = \{ \frac{j}{s} \suchthat j \in J, s \in S \}.\]
  The ideal $J \subseteq A$ is finitely generated, say
  \[ J = A(j_1, \dotsc, j_r).\]
  Then
  \[ \forall j \in J \enspace \exists a_1, \dotsc, a_r \enspace j = \sum_{k=1}^r a_k j_k.\]
  We then have
  \[ \frac{j}{s} = \sum_{k=1}^r \frac{a_k}{s} \cdot \frac{j_k}{1},\]
  and so $I$ is generated by
  \[ \frac{j_1}{1}, \dotsc, \frac{j_r}{1}.\]
\end{proof}

Note that we have used the previous characterization of \cref{noetherian-prop}; indeed, as the submodules of a ring are exactly the ideals.

\begin{proof} (Hilbert basis theorem)
  Let $I \subseteq A[x]$. Write
  \[ J = \{ a \in A \suchthat \exists f \in I \enspace f = ax^n + a_1 x^{n-1} + \dotso \},\]
  that is, $J$ is the set of polynomials with leading term $ax^n$.

  One claims that $J$ is an ideal: indeed, let $j_1, j_2 \in J$. Put
  \[ r = j_1 - j_2.\]
  If $r=0$, we are done; in the other case, proceed.
  Take $f_1, f_2 \in I$ with leading term $f_i = j_i x^{n_i}$.
  Then the leading term of
  \[ x^{n_2} f_1 - x^{n_1} f_2\]
  is
  \[(j_1 - j_2) x^{n_1+n_2}.\]
  Hence, $J$ is an abelian subgroup.

  If $j \in J, a \in A$,
  % TODO
  and so $J$ is an ideal.

  Since $J \subseteq A$ is an ideal of a Noetherian ring, it is finitely generated, say by elements
  \[ j_1, \dotsc, j_r.\]
  One then has
  \[ \exists f_1, \dotsc, f_r \in I\]
  with leading terms of $f_i$ equal to $j_i x^{n_i}$.

  If we let $n = \max(n_i)$, one can modify the $f_i$ to get leading term $f_i = j_i x^n$.

  Let
  \[ SP = I \cap A[x]_{< n}.\]
  This is not an ideal, but an $A$-module; it is isomorphic to $A^{\oplus n}$, and so finitely generated, we write
  \[ SP = Ag_1 + \dotso + Ag_t.\]

  We claim that
  \[ I = (f_1, \dotsc, f_r, g_1, \dotsc, g_t)\]
  as an ideal in $A[x]$.

  Clearly,
  \[ I \supseteq (f_1, \dotsc, f_r, g_1, \dotsc, g_t).\]
  For the other inclusion, we pick $h \in I$ and proceed by induction on degree.

  Write the leading term of $h$ as
  \[ b x^{\deg(h)}.\]
  Now, if $\deg(h) < n$, $h$ must necessarily be an element of $SP$.

  In the other case,
  \[ \deg(h) \geq n\]
  and we can divide with remainder. Write
  \[ b = \sum_{s=1}^r a_s j_s\]
  and consider
  \[ h' = h - \sum a_s f_s x^{\deg(h)-n}.\]
  Then
  \[ \deg(h') < \deg(h) \implies h' \in (f_1, \dotsc, f_r, g_1, \dotsc, g_t)\]
  and so also
  \[ h \in (f_1, \dotsc, f_r, g_1, \dotsc, g_t).\]
  This ends the proof.
\end{proof}


%%% Local Variables:
%%% mode: latex
%%% TeX-master: "../commalg"
%%% End:
