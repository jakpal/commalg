\section{Other notions of dimension for rings}
\begin{df}
  Let \(k \subseteq K\) be fields. We say that \(x_1, \dotsc, x_b \in K\) are \emph{algebraically independent} over \(k\) if there is no nonzero polynomial
  \(0 \neq f \in k[X_1, \dotsc, X_b]\)
  such that \(f(x_1, \dotsc, x_b) = 0.\)
  One says that
  \(\{x_1, \dotsc, x_b\} \subseteq K\) is a \emph{transcendence basis} of \(K\) over \(k\) if \(x_1, \dotsc, x_b\) are algebraically independent and the extension
  \(k(x_1, \dotsc, x_b) \subseteq K\)
  is integral.
\end{df}

\begin{lemma}
  Every two transcendence bases have the same cardinality, provided \(K\) is finitely generated over \(k.\)
\end{lemma}

\begin{df}
  The number of elements in a transcendence basis is called the \emph{transcendence degree} of \(K\) over \(k\) and denoted by
  \(\trdeg_k(K).\)
\end{df}

\begin{theorem}
  Ket \(A\) be a finitely generated \(k\)-algebra. Assume that \(A\) is a domain with \(\fractions(A) = K.\) Then
  \(\dim(A) = \trdeg_k(K).\)
\end{theorem}
\begin{proof}
  Noether normalization implies that there exists a finite extension
  \[k[x_1, \dotsc, x_d] \subseteq A, \quad d = \dim(A).\]
  Let \(S = k[x_1, \dotsc, x_n] \setminus \{0\}.\)
  Then, the extension
  \(K = S^{-1}k[x_1, \dotsc, x_d] \subseteq S^{-1}A\)
  is finite, and because \(S^{-1}A\) is a domain, it follows that \(S^{-1}A = \fractions(A)\) is a field; that is, \(\fractions(S^{-1}A) = \fractions(A).\)

  Hence, we have a finite extension
  \(k(x_1, \dotsc, x_d) \subseteq \fractions(A),\)
  so \(\{x_1, \dotsc, x_d\}\) is a transcendence basis. The claim follows.
\end{proof}

\begin{example}
  \(\dim(k[x,y] / {y^2-x^3-x}) = 1\).
\end{example}


%%% Local Variables:
%%% mode: latex
%%% TeX-master: "../commalg"
%%% End:
