\section{More on modules}
\begin{note}
  In the following, when we write $I \cdot M$, we will mean the linear span of elements of the form $i \cdot m$.
\end{note}

\begin{lemma}[adjugate matrix]
  \label{adjugate matrix}
  Let
  $X \in M_{n \times n}(A)$.
  Then
  \[ \exists Y \in M_{n \times n}(A) \quad \text{such that} \quad Y \cdot X = \diag(d), \enspace d = \det(X).\]
  Concretely,
  $Y$ is given by the formula
  \[ Y \coloneqq [ (-1)^{i+j} \det X_{ji} ]_{ij}, \]
  where
  $X_{ij}$ is the submatrix of $X$ formed by excluding the $i$-th row and the $j$-th column.
\end{lemma}

We skip the proof of \cref{adjugate matrix}.

\begin{theorem}[Cayley-Hamilton]
  Let $M$ be an $A$-module. Let $\phi: M \to M$ be a homomorphism of $A$-modules such that
  \[ \phi(m) = IM = \{ \sum i_k m_k \suchthat i_k \in I, m_k \in M \}. \]
  Then there exist elements
  \[ a_{n-1} \in I, a_{n-2} \in I^2, \ldots, a_0 \in I^n \]
  such that
  \[ \phi^n + a_{n-1} \phi^{n-1} + \ldots + a_0 = 0 = \text{ the zero morphism}. \]
\end{theorem}

\begin{proof}
  \label{cayley-hamilton}
  Let
  \[ \pi \colon A^{\oplus n} \to M \quad \text{via} \quad e_i \mapsto m_i. \]
  Then
  \[ \forall i \enspace \exists u_{ij} \in I \enspace \phi(m_i) = \sum_j u_{ij} m_j\]
  in face of the assumptions.
  Define the lift of $\phi$ to a self-map $\tilde{\phi}$ of the free module $A^{\oplus n}$ by
  \[ \tilde \phi \colon A^{\oplus n} \to A^{\oplus n} \quad \text{via} \quad \tilde \phi(e_i) = \sum_j u_{ij} e_j.\]
  This can be fitted inside a commutative diagram
  \begin{equation*}
    \begin{tikzcd}[row sep=huge, column sep=huge]
      A^{\oplus n} \arrow[swap]{d}{\pi} \arrow{r}{\tilde{\phi}}
      & A^{\oplus n} \arrow{d}{\pi} \\
      M \arrow{r}{\phi}
      & M
    \end{tikzcd}
  \end{equation*}

Let $B \coloneqq A[x]$ and make $A^{\oplus n}$ into a $B$-module by
\[ \forall f \in A^{\oplus n} \quad x . f \coloneqq \tilde{\phi}(f).\]

Now,
\[ X \coloneqq [u_{ij}]_{ij} - \diag(x) \in M_{n \times n} (B).\]
View X as an endomorphism of $A^{\oplus n}$. Then

\[ X . e_i = \sum_j u_{ij} e_j - x \cdot e_i = \sum_j u_{ij} e_j - \tilde{\phi}(e_j) = 0.\]

Hence $X$ acts on $A^{\oplus n}$ as the zero endomorphism.

Now, use \cref{adjugate matrix} to see that
\[ \exists Y \enspace Y \cdot X = \diag(d).\]
Hence
\[ \forall i \enspace 0 = Y \cdot X \cdot e_i = \diag{d} e_i = d \cdot e_i.\]
Expand the determinant by columns:
\[ d = \det(X) = x^n + a_1 x^{n-1} + \dotsc + a_n, \quad a_k \in I^k.\]
So as an endomorphism of $A^{\oplus n}$,
\[\tilde{\phi}^n + a_1 \tilde{\phi}^{n-1} + \dotsc + a_n = 0.\]
But
\[ \phi \circ \pi = \pi \circ \tilde{\phi}.\]
So for all $i$
\begin{align*}
  (\phi^n + a_1 \phi^{n-1} + \dotsc + a_n)(m_i) & = \pi((\tilde{\phi}^n + a_1 \tilde{\phi}^{n-1} + \dotsc + a_n)(e_i)) \\
            & = \pi (0) \\
            & = 0.
\end{align*}
\end{proof}

\begin{lemma}[Nakayama]
  \label{nakayama}
  Let $M$ be an $A$-module, $I \cdot M = M$. If $M$ is finitely generated, then
  \[ \exists i \in I \enspace (1 - i) \cdot M = 0.\]
\end{lemma}
\begin{proof}
  Consider $\id_M \colon M \xto{\simeq} M$.
  By \cref{cayley-hamilton},
  \[ \exists a_i \enspace \id^n + a_1 \id^{n-1} + \dotsc + a_n = 0\]
  and then
  \[ \forall m \in M (1 + a_1 + \dotsc + a_n) \cdot m = 0.\]
  We now let
  \[ -i \coloneqq a_1 + \dotsc + a_n.\]
\end{proof}

\begin{corollary}[local Nakayama]
  \label{local-nakayama}
  Let $A$ be a local ring, $\mathfrak m$ the maximal ideal of $A$, $M$ an $A$-module, $M$ finitely generated, $M = \mathfrak m \cdot M$. Then
  \[ M = 0.\]
\end{corollary}
\begin{proof}
  Apply \cref{nakayama}:
  \[ \exists a \in \mathfrak m \enspace (1-a) \cdot M = 0, \quad (1-a) \notin \mathfrak m.\]
  But $(1 - a)$ is invertible; let its inverse be $b$. Then
  \[ M = 1 \cdot M = b(1-a) M = b \cdot 0 = 0.\]
\end{proof}

\begin{corollary}
  \label{local-nakayama2}
  Let $N \subseteq M$, $M$ finitely generated, $A$ local, $\mathfrak m$ maximal. Let
  \[ M = N + \mathfrak m \cdot M.\]
  Then $M = N$.
\end{corollary}
\begin{proof}
  Write
  \(N + M = M = N + \mathfrak m M\)
  and so
  \(M /{N} = \mathfrak m M /{N}.\)
  By \cref{local-nakayama},
  $M / {N} = 0$.
\end{proof}

\begin{corollary}
  Let $M$ be a finitely generated $A$-module, $(A, \mathfrak m)$ a local ring. Suppose
  \[ m_1, \dotsc, m_k \in M\]
  be such that their images in $M / {\mathfrak m \cdot M}$ generate $M / {\mathfrak m \cdot M}$. Then $m_1, \dotsc, m_k$ generate $M$.
\end{corollary}
\begin{proof}
  Let
  \[ N \coloneqq \sum_{i=1}^k A \cdot m_i \subseteq M.\]
  Then the map
  \[ N \to M / {\mathfrak m \cdot M}\]
  is surjective by assumption, so by \cref{local-nakayama2}:
  \[ M = \mathfrak m \cdot M + N \implies M = N = \sum_{i=1}^k A \cdot m_i.\]
  % Note that
  % $A / {\mathfrak m}$
  % is a field, so
  % $M / {\mathfrak m \cdot M}$
  % is a vector space. This finishes the proof.
\end{proof}

\begin{example}
  Consider $\rational \in \module{\integer}$. Consider $I = (2019) \subseteq \integer$.
  Certainly $\rational = I \cdot \rational$, but there does not exist an $i \in I$ such that
  \[ (1-i) \cdot \rational = 0,\]
  so $\rational$ is not a finitely generated $\integer$-module. 
\end{example}

\begin{example}
  Another counterexample (this time local) is given by $\integer_{(2)}$.
\end{example}

\begin{note}
  Localizing is a typical way of finding non-finitely generated modules.
\end{note}

\begin{example}
  Let $X = S^1$, $A = C(S^1, \real)$. Consider
  \[ \mathfrak m_x = \{ f \suchthat f(x) = 0 \}. \]
  Then $\mathfrak m_x^2 = \mathfrak m_x$, hence $\mathfrak m_x$ is not a finitely generated ideal.
\end{example}

%%% Local Variables:
%%% mode: latex
%%% TeX-master: "../commalg"
%%% End:
