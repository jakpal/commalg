\section{Pictures of spectra}

For any ideal $I \subseteq A$, we get
\[ V(I) = \spec(A/{I}) \hookrightarrow \spec(A)\]
together with a bijection:
\[\{ \bar{\mathfrak p} \subseteq A / I\} \cong \{ \mathfrak q \subseteq A \suchthat \mathfrak q \supseteq I\}.\]
From
\(\pi: A \to A/{I}\)
we get the induced map $\pi^*$, which is injective with image $V(I)$. Hence, we may identify the sets \( \spec(A/{I})\) and \(V(I) \subseteq \spec(A).\)

\begin{df}
  The \emph{maximum spectrum} of a ring R is a subset \(\spec_{\mathrm{max}}(R) \subseteq \spec(R)\) consisting of the maximal ideals of \(R.\)
\end{df}

\begin{example}
  Consider
  \[ \spec(k[x,y]/(xy-1)) \hookrightarrow \spec(k[x,y]) \supseteq \spec_{\mathrm{max}}(k[x,y]) = k^2.\]
A point $(a, b) \in k^2$ is seen as the maximal ideal $(x-a,y-b)$; that comes from $\spec(k[x,y]/{(xy-1)})$ if and only if $ab-1 = 0$.
\end{example}

The upshot is: when $k = \bar k$, $\spec_{max}(k[x_1, \ldots, x_n])$ is well-behaved. We will see later, in the Nullstellensatz, that all maximal ideals in $k[x_1, \ldots, x_n]$ are of the form $(x_1 - a_1, \ldots x_n - a_n)$.

\begin{example}
  $\spec (k[x,y]/{xy})$ gives the "cross" $\{ ab=0 \}$.
\end{example}

\begin{example}
  In the case of
  \[\spec(k[x,y]/(x^2 + y^2 + 1)) \hookleftarrow \spec(k[x,y]),\]
  geometric interpretations transcend our $\real$-intuitions.
\end{example}

\begin{note}
  In general, $\spec_{\text{max}}$ is the set of closed points in $\spec$, and if the ring $A$ be of finite type (that is, finitely generated over a field or $\integer$), then
  \[ \overline{\spec_{\text{max}}(A)} = \spec(A).\]
\end{note}

%%% Local Variables:
%%% mode: latex
%%% TeX-master: "../commalg"
%%% End:
