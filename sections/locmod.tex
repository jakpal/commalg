\section{Localization of modules}

\begin{df}
  The rank of a free module $F$ is
  \[ \rank(F) \coloneqq \dim_{A/{\mathfrak{m}}}(F / {\mathfrak m F}).\]
  This does not depend on the choice of $\mathfrak m$.
\end{df}

\begin{example}
  $ \rank(A^{\oplus I}) = \abs{I}$.
\end{example}

\begin{df}
  Let $S \subseteq A$ be a multiplicative subset, $M \in \module{A}$. The one forms an $S^{-1}A$-module $S^{-1}M$ analogously to $S^{-1}A$ itself. That is,
  \[ S^{-1}M \coloneqq \{ m / {s} \suchthat m \in M, s \in S\}\]
  with
  \[ m / {s} = m' / {s'} \iff \exists t \in S \enspace t(s'm - sm') = 0.\]
\end{df}

\begin{example}
  If
  \( S = \{1, f, f^2, \dotsc \},\)
  then we denote
  \( M_f = S^{-1}M.\)
  Analogously, for
  \( S = A \setminus \mathfrak p,\)
  we will write
  \(M_{\mathfrak p} = S^{-1}M.\)
\end{example}

\begin{df}
  A finitely generated $k$-algebra $A$ is smooth if
  \[ \forall p \in \spec(A) \enspace (\Omega_{A/{k}})_{\mathfrak p} \text{ is free as an $A_{\mathfrak p}$-module}.\]
\end{df}

%Note that for the above definition we need also a definition of localization of modules.

\begin{example}
  $k[x_1, \dotsc, x_n]$ is smooth.
\end{example}

\begin{lemma}
  Let $S \subseteq A$ be a multiplicative subset, $M$ an $A$-module. Then
  \[ S^{-1}A \tensor_A M \cong S^{-1}M\]
  under the map
  \[ (a / {s}) \tensor m \mapsto am /{s}.\]
\end{lemma}
\begin{proof}
  The given map
  \( f \colon S^{-1}A \tensor M \to S^{-1} M\)
  is linear by the universal property of the tensor product. We show that it is an isomorphism.
  \begin{enumerate}
  \item $f$ is ``onto'':
    \[ f(1/{s} \tensor m) = m /{s},\]
  \item $f$ is ``into'': take an element
    \[ n \in S^{-1}A \tensor_A M.\]
    We write
    \[ n = \sum_{k=1}^r a_k/{s_k} \tensor m_k = \sum_{k=1}^r \tilde{a}_k / {s} \tensor \tilde{m}_k.\]
    If we let
    \[s = \prod_{k=1}^r s_k,\]
    we can write
    \[ \sum_{k=1}^r 1 /{s} \tensor \tilde{a}_k m_k = 1 /{s} \tensor (\sum_{k=1}^r \tilde{a}_k m_k) \eqqcolon 1 / {s} \tensor m.\]
    Since any $n$ is of the form, we may proceed as follows. Assume
    $0 = f(n)$;
    we will show
    $n = 0$.
    Have:
    \[ 0 = f(n) = m /{s} \iff \exists t \in S \enspace tm = 0.\]
    Then
    \[ n = 1/ {s} \tensor m = 1 / {(st)} \tensor tm = 1 /{(st)} \tensor 0 = 0.\]
  \end{enumerate}
\end{proof}

\begin{corollary}
  \[S^{-1}( \oplus_{i \in I} M_i) = \oplus_{i \in I} S^{-1} M_i.\]
\end{corollary}
\begin{proof}
  Since the tensor product is left adjoint, it is coexact in that it commutes with colimits (essentially by Yoneda); in particular, with the direct sum.
\end{proof}


%%% Local Variables:
%%% mode: latex
%%% TeX-master: "../commalg"
%%% End:
