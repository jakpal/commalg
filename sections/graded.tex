\section{Graded rings and modules}
Let \((\Lambda, +, 0)\) denote a commutative monoid.

\begin{example}
  \((\naturals, +)\), \((\integer, +)\), \((\integer^n, +)\), \((\integer/{n}, +)\) are all commutative monoids.
\end{example}

\begin{df}
  A \emph{\(\Lambda\)-graded abelian group} \(A\) is a direct sum of abelian groups
  \(\oplus_{l \in \Lambda} A_l\).
  A \emph{\(\Lambda\)-graded ring} is a \(\Lambda\)-graded abelian group such that \(1 \in A_0\) and
  \[\forall l_1, l_2 \in \Lambda \enspace A_{l_1} \cdot A_{l_2} \subseteq A_{l_1 + l_2}.\]
\end{df}

\begin{df}
  For every nonzero element \(a \in A_l\) the \emph{degree} of \(a\) is defined as
  \(\deg(a) \coloneqq l.\)
  The element \(a\) is then called \emph{homogeneous} of degree \(\deg(a).\)
\end{df}

\begin{example}
  The polynomial ring \(k[x]\) is graded by \(\naturals,\) \(\integer\) and \(\integer/{n}\) for any \(n.\)
\end{example}

\begin{df}
  A \emph{\(\Lambda\)-graded module} \(M\) is over a \(\Lambda\)-graded ring \(A\) is a \(\Lambda\)-graded abelian group \(M\) such that
  \(\forall l_1, l_2 \in A \enspace A_{l_1} \cdot M_{l_2} \subseteq M_{l_1 + l_2}.\)
\end{df}

Not every module over a graded ring is a graded module; consider for example a non-homogeneous ideal (that is, one that is not generated by homogeneous elements).

\begin{df}
  A homomorphism of \(\Lambda\)-graded rings is a ring homomorphism \(f \colon A \to B\) such that
  \(\forall l \in \Lambda \enspace f(A_l) \subseteq B_l\).
\end{df}

In the following, we will mostly consider \(\Lambda = \integer, \naturals.\)

\begin{df}
  For \(n \in \integer,\) \(M\) a \(\integer\)-graded module, the \emph{\(n\)-shifted module} \(M(n)\) is defined by
  \(M(n)_d = M_{n+d}.\)
\end{df}

\begin{example}
  The multiplication by \(x\) on \(k[x]\) is not a graded self-homomorphism of \(k[x]\) with the usual grading, but it is a graded homomorphism \(k[x](-1) \to k[x].\)
\end{example}

One may consider the categories of graded groups and graded \(A\)-modules.

\begin{prop}
  Let \(A\) be an \(\naturals\)-graded ring. The following are equivalent:
  \begin{enumerate}
  \item \(A\) is Noetherian,
  \item \(A_0\) is Noetherian and \(A_+ \coloneqq \oplus_{n \geq 1} A_n\) is a finitely generated ideal of \(A\),
  \item \(A_0\) is Noetherian and \(A\) is a finitely generated \(A_0\)-algebra.
  \end{enumerate}
\end{prop}
\begin{proof}
  (\(1 \implies 2\)): \(A_0 = A / {A_+}\) is Noetherian, \(A_+\) is a finitely generated ideal.
  % TODO
  (\(2 \implies 3\)): Let \(A_+ = (f_1, \dotsc, f_r)\) with \(f_i\) homogeneous. We then claim that \(A = A_0[f_1, \dotsc, f_r]\); this is proved by induction on \(n\) with
  \(A_n = (A_0[f_1, \dotsc, f_r])_n.\) Indeed, if we let \(d_i = \deg(f_i),\) then
  \[A_{n+1} = (A_+)_{n+1} = (f_1, \dotsc, f_r)_{n+1} = \sum_{i=1}^k\]
  % TODO
  (\(3 \implies 1\)): Hilbert basis theorem.
\end{proof}

\begin{df}
  Let \(M\) be a finitely generated \(\integer\)-graded \(A\)-module, with \(A\) a Noetherian \(\naturals\)-graded ring and \(A_0 = k.\) Then the \emph{Poincare characteristic} of \(M\) is a function \(\chi_M \colon \integer \to \naturals\) given by
  \(\chi_M(n) = \dim_k M_n.\)
\end{df}

One needs to check that the definition is proper:

\begin{lemma}
  For any \(n,\) the dimension of \(M_n\) is finite.
\end{lemma}
\begin{proof}
  \(M\) is finitely generated and graded, so there exist homogeneous generators \(m_1, \dotsc, m_r \in M\); that is, the map
  \(\oplus_{i=1}^r A(-\deg(m_i)) \to M\)
  is a graded surjection.
  % TODO
  % probably missing assumption
\end{proof}

\begin{example}
  Let \(A = k[x],\) \(M = A.\) Then
  \[\chi_M(n) =
    \begin{cases}
      1, & n \geq 0, \\
      0, & n < 0.
    \end{cases}\]
\end{example}

\begin{example}
  Let \(A = k[x_1, \dotsc, x_d],\) \(M = A.\) If we put \(\deg(x_i) = 1,\) then
  \[\chi_M(n) =
    \begin{cases}
      \binom{d+n-1}{d-1}, & n \geq 0, \\
      \text{polynomial in $n$ of degree $d-1$}, & 
    \end{cases}
  \]
  % TODO literówka
\end{example}

\begin{note}
  If
  \[0 \to V_1 \to V_2 \to \dotso \to V_n \to 0\]
  is an exact sequence of vector spaces over \(k,\) then
  \(\sum_{i=1}^n (-1)^i \dim(V_i) = 0.\)
\end{note}

\begin{corollary}
  \label{cor poincare}
  If
  \[0 \to M_1 \to \dotso \to M_n \to 0\]
  is an exact sequence of graded \(A\)-modules, then
  \(\forall j \enspace \sum_{i=1}^r (-1)^i \dim_k((M_i)_j) = 0.\)
\end{corollary}

\begin{df}
  The \emph{Poincare series} is
  \(P_M(T) = \sum_{n=-\infty}^\infty \chi_M(n) T^n.\)
\end{df}

\begin{note}
  \cref{cor poincare} is then restated as
  \[\sum_{i=1}^r (-1)^i P_{M_i} = 0.\]
\end{note}

\begin{theorem}
  Let \(A = k[x_1, \dotsc, x_r],\) \(\deg(x_i) = d_i \geq 1.\)
  Let \(M\) be a finitely generated \(A\)-module.
  Then
  \[P_M = \frac{W}{\prod_{i=1}^r (1 - T^{d_i})}, \quad W \in \integer[T].\]
\end{theorem}
\begin{proof}
  % exam
  Induction on \(r\). In \(r = 0,\) let \(A = k\), \(M\) a finitely generated vector space. For large enough \(n,\) \(M_n = 0\) and so \(P_M\) is a polynomial in \(T.\)

  For the induction step, consider
  \[0 \to K \to M(-d_r) \xto{x_r} M \to M / {x_r M} \to 0,\]
  \(K = \ker(x_r).\)
  \(x_r \cdot K = 0,\) so \(K\) is a finitely generated \(k[x_1, \dotsc, x_{r-1}]\)-module.
  One then has
  % TODO why
  \(P_K - P_{M(-d_r)} + P_M - P_{M /{x_r M}} = 0\)
  and hence
  \(P_{M(-d_r)} = \sum \chi_{M(-d_r)}(n)T^n = \sum \chi_M(n-d_r) T^n = P_M(1-T^{d_r}) = P_{M/{x_r M}} - P_K.\)
  Thus,
  \(P_m = \frac{1}{1-T^{d_r}}(P_{M/{x_r M} - P_K}).\)
\end{proof}

\begin{corollary}
  If \(d_1 = d_2 = \dotsb = d_r = 1,\) then
  \[\exists h_M \in \rational[t] \enspace \forall n >> 0 \enspace h_M(n) = \chi_M(n).\]
\end{corollary}
\begin{proof}
  Combinatorics.
\end{proof}

% skipped some scratch

\begin{prop}[graded Nagata trick]
  Let \(k\) be algebraically closed, \(\deg(x_i) = 1,\) \(f \in k[x_1, \dotsc, x_r]\) a homogeneous element. Then there exist homogeneous elements
  \(x_1', \dotsc, x_{r-1'}\)
  such that
  \(k[x_1', \dotsc, x_{r-1}', f] \subseteq k[x_1, \dotsc, x_r]\) is a finite extension.
\end{prop}

\begin{corollary}[graded Noether normalization]
  If \(A = k[x_1, \dotsc, x_N] / {I}\) is graded, then there exist homogeneous elements \(y_1, \dotsc, y_d \in A\) of degree \(1\) such that
  \(k[y_1, \dotsc, y_r] \subseteq A\) is finite, \(d = \dim(A).\)
\end{corollary}
\begin{proof}
\cite{eisenbud1995commutative}
\end{proof}


\begin{theorem}
  Let \(k\) be algebraically closed.
  \(P_A = \frac{W}{(1-T)^d},\) where \(d = \dim(A).\) Consequently, \(\deg(h_A) = d - 1.\)
\end{theorem}
\begin{proof}
  Fix \(y_1, \dotsc, y_d \in A\) as above, \(S = k[y_1, \dotsc, y_d] \subseteq A\) a graded submodule such that \(A\) is a finitely generated \(S\)-module.

  For all \(n\), one then obtains
  \[\sum_{i=1}^N \chi_S(n - e_i) \geq \chi_A(n) \geq \chi_S(n).\]
  It follows that \(\deg(h_A) = \deg(h_S) = d-1.\)
  By arguments combinatorial, the claim follows.
\end{proof}

We will consider an action of \(k^*\) on \(M = \oplus_{i \in \integer} M_i\) by \(t . m_ = t^{-i} m\) when \(m \in M_i.\)

\begin{prop}[homogeneous Nullstellensatz]
  \label{homogeneous Nullstellensatz}
  Let \(k\) be algebraically closed, \(S = k[x_1, \dotsc, x_r]\) be graded with \(\deg(x_r) = 1,\) \(\rad{I} = I \subseteq S.\) Then the following are equivalent:
  \begin{enumerate}
  \item \(I\) is homogeneous,
  \item \(k^* \cdot I \subseteq I,\)
  \item \(V = V_{max}(I),\) \(k^* \cdot V \subseteq V.\)
  \end{enumerate}
\end{prop}

Consider the action of \(k^*\) on \(k^n\) by \(t \cdot (v_1, \dotsc, v_n) = (tv_1, \dotsc, tv_n).\) Moreover, denote
\(\mathrm{Proj} (S) = \{ \mathfrak p \subseteq S\} \setminus S_+.\)

\begin{corollary}
  Maximal elements of \(\mathrm{Proj}(S)\) are the closures of orbits of \(k^*\) on \(k^n\), that is, lines through \(0\) in \(k^n,\) which is the same as the projective space \(k \mathbb P^{n-1}.\)
\end{corollary}

\begin{lemma}
  \label{homogeneous ideal tfae}
  Let \(A\) be a graded ring, \(I \subseteq A\) an additive subgroup (resp. an ideal). The following are equivalent:
  \begin{enumerate}
  \item \(I\) is generated as an additive subgroup (respectively, as an ideal) by homogeneous elements,
  \item \(\forall i \in I \enspace i = \sum_j i_j, \enspace i_j \in A_j.\)
  \end{enumerate}
\end{lemma}

\begin{df}
  If \(I\) satisfies the conditions of \cref{homogeneous ideal tfae}, it is called \emph{homogeneous}.
\end{df}

\begin{note}
  If \(I, J\) are homogeneous, then so are \(I + J,\) \(I \cap J\) and many other ``usual'' constructions on ideals.
\end{note}

\begin{lemma}
  Let \(A\) be an \(\naturals\)-graded \(k\)-algebra, \(k\) infinite. Let \(I \subseteq A\) be a vector subspace. Then the following are equivalent:
  \begin{enumerate}
  \item \(I\) is homogeneous,
  \item \(k^*. I \subseteq I,\) where \(k^*\) acts by the \emph{torus action} \(t.a_i = t^{-i} \cdot a_i\) for \(a_i \in A_i.\)
  \end{enumerate}
\end{lemma}

\begin{proof}
  For the implication \(1 \implies 2,\) take \(a \in I,\) \(a_i = \sum a_j,\) \(a_j \in I_j.\) Then
  \[t.a = t.\sum a_j = \sum t^{-j}a_j.\]
  For the converse, take \(a \in A\) with \(a = \sum_{j=0}^r a_j,\) \(a_j \in A_j.\) For any \(t \in k^*,\) one has \(t.a \in I.\) We choose \(r\) different elements
  \(t_0, \dotsc, t_r \in k^*;\) this is possible because \(k\) is infinite.
  We consider the action of the Vandermonde matrix \(\mathrm{Vdm}(t_0^{-1}, \dotsc, t_r^{-1})\) on the vector \((a_0, \dotsc, a-r).\) The Vandermonde matrix is invertible in the vector space \(A/{I}\) and at the same time, it acts on \((a_0, \dotsc, a_r)\) by zero. Hence, we get the condition \(2,\) as it is equivalent to every \(a_i\) being zero in \(A/{I}.\)
\end{proof}

\begin{corollary}[of \cref{homogeneous Nullstellensatz}]
  The standard Nullstellensatz bijection for \(S = k[x_1, \dotsc, x_n]\) restricts to
  \[\{I = \rad{I} \subseteq S \text{ homogeneous}\} \cong \{k^*\text{-stable algebraic subsets}\}.\]
\end{corollary}

% TODO shuffle theorems in this section - two lectures, one subject

For the following note that the \(k^*\)-action on \(k^n\) is by coordinate multiplication, while on \(S\) it is the torus action.

\begin{proof}(of \cref{homogeneous Nullstellensatz})
  Let \(V = V_{\text{max}}(I).\) We begin by showing \(2 \implies 1.\) Assume \(k^*. V \subseteq V.\) Lines through \(0\) are then closures of \(k^*\)=orbits in \(k^n.\) \(V\) is the sum of all lines in \(V\) through \(0.\) Apply Nullstellensatz to get
  \[I = I(V) = \cap \{I(l) \suchthat l \subseteq V \text{ a line}\}.\]
  We now claim that for any line \(l\) through \(0,\) \(I(l)\) is homogeneous. Indeed, if we take \((\alpha_1, \dotcs, \alpha_n) \in l,\) \(\alpha_i \neq 0,\) then
  \(I(l) = (\alpha_1 x_i - x_1 \alpha_i)_{i = 2, \dotsc, n}\)
  is homogeneous as it is generated by homogeneous elements.
  \(I\) is then the intersecton of homogeneous ideals and so it is homogeneous.

  We now show \(1 \implies 2.\) Let \(f \in S,\) \(t \in k^*,\) \(\alpha \in k^n\). Then the actions of \(k^*\) on \(k^n\) and on \(S\) are related by
  \[f(t^{-1}.\alpha) = (t.f)(\alpha).\]
  % TODO could write out, don't really want to
  We claim that \(\alpha \in V_{\text{max}}(I)\) if and only if \(\forall f \in I \enspace f(\alpha) = 0.\)
  % TODO some equalities inbetween
  This is further equivalent to
  \(t^{-1}.\alpha \in V_{\text{max}}(I).\)
\end{proof}

\begin{note}
  Closed subsets of \(k^n \setminus 0 /{k^*} = k \mathbb P^{n-1}\) with the quotient topology correspond to the \(k^*\)-stable closed subsets of \(k^n \setminus 0.\) This is not the same as the \(k^*\)-stable algebraic subsets of \(k^n,\) as that space is missing \(\{0\}.\)

  The space \(k \mathbb P^{n-1}\) is Hausdorff, unlike the affine spaces.
\end{note}

















%%% Local Variables:
%%% mode: latex
%%% TeX-master: "../commalg"
%%% End:
