\section{Krull's principal ideal theorem}

\begin{example}
  Consider the elliptic curve
  \(V(y^2-x^3-x) \subseteq \spec(k[x, y]).\)
  The Jacobian has constant rank \(1\) (on the zero locus).
  % TODO so what?
\end{example}

We will aim to show existence of regular points. We recall a fact.
\begin{note}
  If \(f \colon A \to B\) is a ring homomorphism, \(M\) is an \(A\)-module and \(N\) a \(B\)-module, then there is a natural bijection
  \[\hom_A(M, N) \cong \hom_B(B \tensor_A M, N)\]
  that generalizes to an adjunction between tensoring by \(B\) and forgetting the structure of a \(B\)-module.
\end{note}

\begin{theorem}[Krull's principal ideal theorem]
  \label{krull's principal ideal theorem}
  Let \(A\) be Noetherian, \(f \in A,\) \(\mathfrak p \in V(f)\) minimal. Then \(\dim(A_{\mathfrak p}) \leq 1.\)
\end{theorem}
\begin{proof}
  Let \(\mathfrak q \subseteq \mathfrak p\) be distinct prime ideals. We wish to have
  \(\spec(A_{\mathfrak q}) = \{ \mathfrak q A_{\mathfrak q}\}.\)
  We replace \(A\) by \(A_{\mathfrak p}\) and write further on \(A\) instead of \(A_{\mathfrak p}.\)
  Now, because \(\mathfrak p\) is minimal over \(f,\) we have
  \(V(f) = \{\mathfrak p A_{\mathfrak p}\}.\)
  Because \(\dim(A/{(f)}) = 0,\) \(A/{(f)}\) is Artinian.
  Let \(\pi \colon A \to A_{\mathfrak q}.\) Now, a sequence
  \[\mathfrak q A_{\mathfrak q} \supseteq \mathfrak q^2 A_{\mathfrak q} \supseteq \dotso \supseteq \mathfrak q^n A_{\mathfrak q}\]
  leads to a sequence
  \[I_1 \coloneqq \pi^{-1}(\mathfrak q A_{\mathfrak q}) \supseteq\]
  % TODO etc.
  Then \(I_n+(f)/{(f)}\) stabilizes, because \(A/{(f)}\) is Artinian. Hence, \(I_n + (f)\) stabilizes. In particular, there exists an \(n\) such that \(I_n + (f) = I_{n+1} + (f).\)
  Hence, if we choose \(i \in I_n,\) there will be an \(\tilde{i} \in I_{n+1}\) and an \(a \in A\) such that
  \(i = \tilde{i} + af.\)

  We deduce that \(I_n = I_{n+1} + f \cdot I_n\). By Nakayama,
  % TODO ref
  we get \(I_n = I_{n+1}.\)
  Since we have
  \[\mathfrak q^n A_{\mathfrak q} = \pi(I_n)A_{\mathfrak q} = \pi(I_{n+1}) A_{\mathfrak q} = \mathfrak q^{n+1} A_{\mathfrak q},\]
  another use of Nakayama implies that \(\mathfrak q^n A_{\mathfrak q} = 0\) and so \(\mathfrak q A_{\mathfrak q}\) is nilpotent.
  This finishes the proof.
\end{proof}

\begin{corollary}
  \label{krull corollary}
  Let \(A\) be Noetherian,
  \(f_1, \dotsc, f_r \in A,\)
  \(\mathfrak p \in V(f_1, \dotsc, f_r)\) minimal.
  Then
  \(\dim(A_{\mathfrak p}) = 1.\)
\end{corollary}
\begin{proof}
  We proceed by induction on \(r,\) seeing that the case \(r=1\) is correct. In the following, replace \(A\) by \(A_{\mathfrak p}.\)

  For the induction step, consider a proper inclusion \(\mathfrak q \subseteq \mathfrak p\) of prime ideals, with no prime ideals between. We claim that \(\dim(A_{\mathfrak q}) \leq r-1.\)
  We have
  \(\{f_1, \dotsc, f_r\} \not\subseteq \mathfrak q,\)
  for example \(f_1 \notin \mathfrak q.\)

  Then \(\mathfrak p \in V(\mathfrak q + (f_1))\) is a  minimal element; if not, then \(\mathfrak q = \mathfrak r \subseteq \mathfrak p\) is a proper inclusion, \(f_1 \in \mathfrak p.\)

  Now,
  \[\spec(A/{(\mathfrak q + (f_1))}) = \{\mathfrak p\} \implies \nil(A / {\mathfrak q + (f_1)}) = \mathfrak p\]
  and so \(f_2, \dotsc, f_r \in \nil(A/{\mathfrak q + (f_1)}),\) hence
  \[\exists n \forall i = 2, \dotsc, r \enspace \exists g_i \in \mathfrak q \enspace \exists a_i \in A \enspace f_i^n = g_i + f_1 a_i.\]
  Consider now \(\tilde{f}_1 \in A/{(g_2, \dotsc, g_r)},\) and \(\tilde{\mathfrak p}\) a minimal element of \(V(f_1).\)

  By \cref{krull's principal ideal theorem}, we get
  \(\dim(A/{(g_2, \dotsc, g_r)}) \leq 1,\)
  \(\spec(A/{(g_2, \dotsc, g_r)}) = \{\tilde{\mathfrak q}, \tilde{\mathfrak p}\},\)
  with \(\tilde{\mathfrak q}\) a minimal element of the spectrum.

  \(\mathfrak q \in V(g_2, \dotsc, g_r)\) is minimal, so by induction \(\dim(A_{\mathfrak q}) \leq r - 1.\) Then \(\dim(A_{\mathfrak p}) \leq r.\)
\end{proof}

\begin{corollary}
  Let \(A\) be Noetherian, \(\mathfrak m\) a maximal ideal. Then
  \[\dim(A_{\mathfrak m}) \leq \dim_{A/{\mathfrak m}}(\mathfrak m /{\mathfrak m^2}).\]
\end{corollary}
\begin{proof}
  Let \(r = \dim_{A/{\mathfrak m}}(\mathfrak m /{\mathfrak m^2}).\) Choose \(f_1 \dotsc, f_r\) that span \(\mathfrak m /{\mathfrak m^2}\). Nakayama implies that
  \(\mathfrak m A_{\mathfrak m} = (f_1, \dotsc, f_r) A_{\mathfrak m},\)
  \(\mathfrak m A_{\mathfrak m} \in V(f_1, \dotsc, f_r),\)
  and we use the \cref{krull corollary} to derive the result.
\end{proof}

\begin{note}
  In the above,
  \[\mathfrak m /{\mathfrak m^2} = \mathfrak m A_{\mathfrak m} /{\mathfrak m^2 A_{\mathfrak m}}.\]
\end{note}









%%% Local Variables:
%%% mode: latex
%%% TeX-master: "../commalg"
%%% End:
