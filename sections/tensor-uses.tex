\section{Why do we care?}


First of all, as noted before, we have the folowing isomorphisms:
\[ M \tensor_A A /{I} \cong M /{IM},\]
\[ M \tensor_A S^{-1}A \cong S^{-1} M.\]

Moreover:
\begin{prop}
Tensoring with an $A$-algebra $B$ is left adjoint to the forgetful functor
\[ \module{B} \to \module{A}.\]
\end{prop}

\begin{prop}
  \label{tensor-coproduct-of-algebras}
Suppose $A$-algebras $B, C$ are given. Then
\[ B \tensor_A C\]
is an $A$-algebra with multiplication
\[ (b_1 \tensor c_1) \cdot (b_2 \tensor c_2) = (b_1 b_2) \tensor (c_1 c_2).\]
We claim that
\[ B \tensor_A C\]
is in fact the coproduct of $B$ and $C$.
\end{prop}

\begin{note}
In the same setting, one has the diagram
\begin{equation*}
  \begin{tikzcd}
    & \spec(B) \arrow{dl} & \\
    \spec(A)
    & & \spec(B \tensor_k C) \arrow{ul} \arrow{dl} \\
    & \spec(C) \arrow{ul} &
  \end{tikzcd}
\end{equation*}
Note that then
\[ \spec(B \tensor_k C) \neq \spec(B) \times \spec(C);\]
we only get the arrows demanded that make the diagram commute.
\end{note}

We note the following special cases of \cref{tensor-coproduct-of-algebras}.

\begin{corollary}
  Let $A = k$ a field,
  \[ B = k[x_1, \dotsc, x_s], \quad C = k[y_1, \dotsc, y_k].\]
  Then
  \[ B \tensor_k C \cong k[x_1, \dotsc, x_s, y_1, \dotsc, y_t].\]
\end{corollary}
\begin{proof}
  Note that $B$ and $C$ are free $k$-modules, so $B \tensor_k C$ is a free $k$-module with basis
  \[x_1^{a_1} \tensor \dotso \tensor x_s^{a_s} \tensor y_1^{b_1} \tensor \dotso \tensor y_t^{b_t}.\]
  One checks that the obvious map defined by the universal property is an isomorphism.
\end{proof}

\begin{corollary}
  Suppose
  \[ B = k[x_1, \dotsc, x_s] / {I},\]
  \[ C = k[y_1, \dotsc, y_t] /{J}.\]
  Then
  \[ B \tensor_k C \cong k[x_1, \dotsc, x_s, y_1, \dotsc, y_t] / {(I + J)}.\]
\end{corollary}

In differential geometry, the fiber of a map $f \colon X \to Y$ over a point is not typically a manifold.

In contrast, in the $\spec$-world, we have the following.

\begin{df}
  Let $f \colon A \to B$ be a homomorphism, $\mathfrak m \subseteq A$ a maximal ideal. Then the fiber of $f$ over $\mathfrak m$ is
  \[ B /{\mathfrak m B}.\]
\end{df}

\begin{prop}
  In the above setting,
  \[\spec(B /{\mathfrak m B}) \cong \{ \mathfrak p \subseteq B \suchthat f^*(\mathfrak p) = \{ \mathfrak m \} \}.\]
\end{prop}

Note that
\[ \spec(A_{\mathfrak p}) = \{ \mathfrak q \subseteq \mathfrak p\}, \quad
\spec(A /{\mathfrak p}) = \{ \mathfrak q \supseteq \mathfrak o\}.\]

\begin{df}
  More generally, if $f \colon A \to B$ is a homomorphism, $\mathfrak p \subseteq B$ a prime ideal, then the fiber of $f$ over $\mathfrak p$ is given by
  \[ B \tensor_A \kappa(\mathfrak p),\]
  where
  \[ \kappa(\mathfrak p) = A_{\mathfrak p} / {\mathfrak p A_{\mathfrak p}}.\]
\end{df}

\begin{prop}
  For any prime ideal $\mathfrak p \subseteq B$, $\kappa(\mathfrak p)$ is a field.
\end{prop}

\begin{note}
  If $\mathfrak m$ is maximal, then
  \[ \kappa(\mathfrak m) \cong A / {\mathfrak m}.\]
\end{note}

\begin{prop}
  There is a commutative diagram
  \begin{equation*}
    \begin{tikzcd}[row sep = huge, column sep = huge]
      \spec(B) \arrow{d}{f^*}
      & \spec(B \tensor_A \kappa(\mathfrak p)) \arrow{l} \arrow{d} \\
      \spec(A) &
      \spec(\kappa(\mathfrak p) \arrow{l}
      &
    \end{tikzcd}
  \end{equation*}
\end{prop}




%%% Local Variables:
%%% mode: latex
%%% TeX-master: "../commalg"
%%% End:
