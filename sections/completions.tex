\section{Completions}
Consider the singular curve \(\spec(A),\) \(A = k[x,y]/{x^3+x^2-y^2}.\) This has the normalization \(A \to \tilde{A} = k[t];\) there is the induced map of spectra
\(\mathbb{A}^1 = \spec(\tilde{A}) \to \spec(A),\) which is an isomorphism outside the singular point \((0, 0).\)

One should note that the localization \(A_{\mathfrak{m}}\) at the singular point is still a domain - in that, it can be said it does not recognize the components it posesses. On the other hand, the local completion we aim to define will not be a domain anymore.

\begin{df}
  A \emph{topological (abelian) group} is a group (an abelian group) with group maps \(\mu \colon G \times G \to G,\) \(\mathrm{inv} \colon G \to G\) together with a topology in which \(\mu\) and \(\mathrm{inv}\) are continuous.
\end{df}

\begin{note}
  For all \(g \in G,\) the map \(\mu_g \colon G \to G,\) \(h \mapsto g + h\) is a homeomorphism with inverse \(\mu_{-g}.\)
\end{note}

\begin{corollary}
  The topology on \(G\) is determined by the set of neighbourhoods of \(0 \in G.\)
\end{corollary}

\begin{df}
  The completion of an abelian topological group \(G\) is \(\tilde{G}\) - the quotient
  \[\text{Cauchy sequences}/{\text{sequences converging to 0}}.\]
\end{df}

\begin{example}
  The completion of \(\rational\) (with the Euclidean topology) is \(\real.\)
\end{example}

\begin{example}
  The completion of \(\rational\) with the \(p\)-adic topology is \(\rational_p.\)
\end{example}

\begin{example}
  Let \(G = G_0 \supseteq G_1 \supseteq \dotso\) be a sequence of subgroups. Then one can defined a topology on \(G\) by taking \(G_n\) as neighbourhoods of \(0.\)

  In particular, let \(A\) be a ring, \(I\) an ideal, \(G = (A, +),\) \(G_n = I^n.\) In this case, one can consider \(\tilde{A}_I.\)

  In the case that \(A = k[x_1, \dotsc, x_n],\) \(\tilde{A}_I = k[x_1, \dotsc, x_n],\) \(I = (x_1, \dotsc, x_n).\)
\end{example}

\begin{prop}
  If \(G = G_0 \supseteq G_1 \supseteq \dotso\) and we consider the system made up of maps
  \(G/{G_{n+1}} \xto{\pi_n} G/{G_n},\) then
  \[\lim_n G/{G_n} = \{a_n \in G/{G_n} \suchthat \pi_{n+1}(a_{n+1}) = a_n\} \subseteq \prod_n G/{G_n}.\]
\end{prop}

\begin{note}
  If \(G/{G_n}\) are compact, then \(\prod G/{G_n}\) is compact by Tychonoff and so is \(\lim G/{G_n}.\)
\end{note}

\begin{prop}
  There is a unique isomorphism \(\tilde{G} \to \lim G/{G_n}\) that commutes with maps \(G \to \tilde{G}\) and \(G \to \lim G/{G_n},\) \(g \mapsto (g).\)
\end{prop}
\begin{proof}
  Let \((a_n) \in \tilde{G}.\) Since \((a_n)\) is Cauchy, we get a homomorphism
  \(\tilde{G} \to G/{G_m}\) sending \((a_n)\) to \(a_n\) modulo \(G_m.\)
  In this, \((a_n)\) is mapped to \(0 \in \lim G/{G_n}\) if and only if \((a_n)\) converges to \(0.\) This shows that the map \(\tilde{G} \to \lim G/{G_n}\) is injective.

  For surjectivity, one chooses \(\tilde{g}_n \in \lim G/{G_n}\) and any \(g_n\) such that \(g_n = \tilde{g}_n\) modulo \(G_n.\) Then \(g_n\) is Cauchy.
\end{proof}

\begin{corollary}
  The completion of \(k[x_1, \dotsc, x_r]_{\mathfrak{m}}\) is
  \[\lim_n k[x_1, \dotsc, x_r]/{(x_1, \dotsc, x_r)^n} = k[[x_1, \dotsc, x_r]].\]
\end{corollary}

More generally, if one fixes \((A, I)\), then for any \(A\)-module \(M\) there exists a topology given by \((I^nM)_n,\) as well as a completion \(\tilde{M},\) which is an \(\tilde{A}\)-module.

For \(f \colon M \to N\) an \(A\)-module homomorphism, one has \(g(I^nM) \subseteq I^nM,\) and so one gets \(\tilde{f} \colon \tilde{M} \to \tilde{N}.\)

\begin{theorem}
  The functor \(M \to \tilde{M}\) is exact on finitely generated \(A\)-modules, provided \(A\) is Noetherian.
\end{theorem}
\begin{proof}
  Can be found in \cite{atiyah1994introduction}.
\end{proof}

\begin{example}
  In the case \(A = k[x, y],\) \(J = (x^3 + x^2 - y)\) one gets
  \(\tilde{(A/{J})} = \tilde{A}/{\tilde{J}} = k[[x, y]] /{\tilde{J}}.\)
\end{example}

\begin{theorem}[Cohen]
  Let \((A, \mathfrak{m})\) be Noetherian, \(I = \mathfrak{m},\) \(A\) contains a field. Then there exists a ring homomorphism \(A/{\mathfrak{m}} \into A\) and an isomorphism
  \(\tilde{A} \cong A/{\mathfrak{m}}[[x_1, \dotsc, x_r]]/{I}.\)
  If \(A\) is regular, then \(I = 0\) and \(\tilde{A} \cong A/{\mathfrak{m}}[[x_1, \dotsc, x_r]].\)
\end{theorem}


%%% Local Variables:
%%% mode: latex
%%% TeX-master: "../commalg"
%%% End:
