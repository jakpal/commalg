\section{Finite and integral ring extensions}
Consider $B$ an $A$-algebra.

\begin{df}
  An element $b \in B$ is \textit{integral} over $A$ if
  \[ \exists a_{n-1}, \dotsc, a_0 \in A \enspace b^n + a_{n-1} b^{n-1} + \dotso + a_0 = 0.\]

  $B$ is integral over $A$ if all its elements are integral over $A$.
\end{df}

\begin{df}
  An $A$-algebra $B$ is \textit{finite} over $A$ if it is finitely generated as an $A$-module, that is,
  \[ \exists b_1 \dotsc, b_k \enspace B = A b_1 + \dotso + A b_k.\]
\end{df}

\begin{example}
  The following hold:
  \begin{enumerate}
  \item $\integer \to \integer[i]$ is finite,
  \item $\integer \to \integer /{n}$ is finite,
  \item more generally, $A \to A/{I}$ is finite,
  \item $\integer \to \integer[x]$ is not finite.
  \end{enumerate}
\end{example}

\begin{lemma}
  If $A \to B$ is finite, then it is integral.
\end{lemma}
\begin{proof}
  Let $b \in B$ and consider the multiplication map
  \[ \phi \colon B \to B, \quad \phi(a) = a \cdot b.\]
  This is an $A$-module homomorphism.
  By Cayley-Hamilton with $I = (1)$, one has
  \[ \exists a_{n-1}, \dotsc, a_0 \in A \enspace b^n + a_{n-1} b^{n-1} + \dotso + a_0 = 0.\]
\end{proof}

\begin{lemma}
  \label{lem-integral-fingen-fin}
  If $A \to B$ is integral and $B$ is a finitely generated $A$-algebra, then $A \to B$ is finite.
\end{lemma}
\begin{proof}
  Let
  \[ B = A[x_1, \dotsc, x_n]/{I}.\]
  Then every $x_i \in B$ is integral over $A$ and so
  \[ \exists n \enspace \forall i \enspace \exists a_0^{(i)} \enspace x_i^n + a_{n-1}^{(i)} + \dotso + a_0^{(i)} = 0.\]
  We check that $B$ is generated as an $A$-module by the finitely many monomials
  \[ \{ x_1^{c_1}, \dotsc, x_n^{c_n}, 0 \leq c_i \leq n\}.\]
\end{proof}

\begin{example}
  The extension
  \[ \rational \to \bar \rational\]
  is integral, but not finite.
\end{example}

\begin{df}
  For $B$ and $A$-algebra, $b \in B$ an element, $A[b]$ will denote the smallest $A$-algebra contained in $B$ and containing $b \in B$.
\end{df}

\begin{prop}
  \label{prop-integral}
  Let $A \to B$ be an $A$-algebra, $b \in B$. The following conditions are equivalent:
  \begin{enumerate}
  \item $v$ is integral over $A$,
  \item $A[b]$ is a finitely generated $A$-module,
  \item $A[b]$ is contained in a finitely generated $A$-module
  \end{enumerate}
\end{prop}
\begin{proof}
  ``$1 \implies 2 \implies 3$'' is formal: the first one uses the same trick as in \cref{lem-integral-fingen-fin}.

  For ``$ 3 \implies 1$, fix a finitely generated $A$-module $A[b] \subseteq C$. Then the multiplication map
  \[ \phi_b \colon C \to C\]
  is an $A-$module homomorphism, so Cayley-Hamilton implies
  \[ \exists n \exists a_{n-1}, \dotsc, a_0 \in A \enspace b^n + a_{n-1} b^{n-1} + \dotso + a_0 = 0.\]
\end{proof}

\begin{df}
  Let $B$ be an $A$-algebra. Then the \textit{integral closure} of $A$ in $B$ is
  \[ \bar A = \{ b \in B \suchthat \text{$b$ is integral in $A$}\}.\]

  The \textit{normalization} of a domain $A$ is its integral closure in the field of fractions $\text{Frac}(A)$.
\end{df}


\begin{corollary}
  $\bar A$ is an $A$-algebra.
\end{corollary}
\begin{proof}
  Consider $x, y \in \bar A$. By \cref{prop-integral}, $A[x]$ and $A[y]$ are finitely generated $A$-modules. In fact, by the proof of \cref{lem-integral-fingen-fin}, $A[x, y]$ is a finitely generated $A$-module.

  One has
  \[ A[x - y] \subseteq A[x,y], \quad A[xy] \subseteq A[x, y],\]
  so by \cref{prop-integral}, point 3,
  \[ x- y, xy \in \bar A.\]
  Hence, $\bar A$ is a ring. One has also the map
  \[ A \to \bar A\]
  which makes $\bar A$ an $A$-algebra; indeed, an $A$-subalgebra of $B$.
\end{proof}

\begin{example}
  Let $\integer \to \rational \to K$ with $\rational \to K$ finite.
  Then one has also
  \[ O_K = \bar Z \to K\]
  - the ring of algebraic integers in $K$. This is Noetherian.
\end{example}

\begin{theorem}[Nagata]
  If $A$ is finitely generated over $k$, then the normalization of $A$ is as well. Moreover,
  \[ A \to \bar A\]
  is finite.
\end{theorem}









%%% Local Variables:
%%% mode: latex
%%% TeX-master: "../commalg"
%%% End:
