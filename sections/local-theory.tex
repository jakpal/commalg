\section{Local theory}
One wonders what the dimension of \(k[x_1, \dotsc, x_n] / {(f_1, \dotsc, f_r)}\) may be.

For instance, \(\dim(k[x_1, \dotsc, x_n]/{(f)})\) may be \(n\) if \(f=0;\) \(-1\) if \(f \in k^*\); \(n-1\) otherwise.

In general, \(\dim(k[x_1, \dotsc, x_n]/{(f_1, \dotsc, f_r)})\) may be \(-1\) if \((f_1, \dotsc, f_r) = (1)\) and otherwise it is no smaller than \(n - r.\)

\begin{example}
  \[\dim(k[x_1, x_2, x_3]/{(x_1^2, x_1 x_2, x_2^2)}) = \dim(k[x_1, x_2, x_3]/{(x_1, x_2)}) = 1.\]
\end{example}

\begin{theorem}[Krull's principal ideal theorem]
  Let \(A\) be a Noetherian ring, \(f \in A.\) If \(\mathfrak p \in V(f)\) is minimal, then
  \(\dim(A_{\mathfrak p}) \leq 1.\)
\end{theorem}

\begin{corollary}
  Let \(A\) be a Noetherian ring, \(f_1, \dotsc, f_r \in A.\) If \(\mathfrak p \in V(f_1, \dotsc, f_r)\) is minimal, then \(\dim(A_{\mathfrak p}) \leq r.\)
\end{corollary}

\begin{corollary}
  Let \(\mathfrak m \subseteq A\) be a  maximal ideal. Then
  \[\dim(A_{\mathfrak m}) \leq \dim_{A/{\mathfrak m}} (\mathfrak m /{\mathfrak m^2}).\] 
\end{corollary}

\begin{df}
  A maximal ideal \(\mathfrak m \in \spec(A)\) is \emph{regular} if
  \[\dim(A_{\mathfrak m}) = \dim_{A/{\mathfrak m}} (\mathfrak m /{\mathfrak m^2}).\]
\end{df}

\begin{theorem}
  If \(\mathfrak m\) is regular, then \(A_{\mathfrak m}\) is a unique factorization domain (UFD).
\end{theorem}
\begin{proof}
  See Eisenbud.
\end{proof}

\begin{note}
  Being regular is a property of \(A_{\mathfrak m}.\)
\end{note}

\begin{example}
  All DVRs are regular.
\end{example}

\begin{note}
  By Nakayama, \(\dim_{A/{\mathfrak m}}(\mathfrak m/{\mathfrak m^2})\) is equal to the number of generators of \(\mathfrak m A_{\mathfrak m},\) provided \(A\) is Noetherian.

  Indeed, suppose that \(f_1, \dotsc, f_d \in \mathfrak m A_{\mathfrak m}\) are such that \(\tilde{f}_1, \dotsc, \tilde{f}_d\) span \(\mathfrak m /{\mathfrak m^2} = \mathfrak m A_{\mathfrak m}/{\mathfrak m^2 A_{\mathfrak m}}.\)

  Then,
  \[(f_1, \dotsc, f_d) A_{\mathfrak m} + \mathfrak m^2 A_{\mathfrak m} = \mathfrak m A_{\mathfrak m}\]
  and by Nakayama, \((f_1, \dotsc, f_d) A_{\mathfrak m} = \mathfrak m A_{\mathfrak m}.\)
\end{note}






%%% Local Variables:
%%% mode: latex
%%% TeX-master: "../commalg"
%%% End:
