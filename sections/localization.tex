\section{Localization}
\begin{df}
Let $A$ be a ring. A subset $S \subseteq A$ is called \textit{multiplicative} if the following conditions hold:
\begin{enumerate}
\item $1 \in S$,
\item $\forall s, t \in S \enspace st \in S$.
\end{enumerate}
\end{df}

% TODO
% The localization is a universal ??? which inverts ???.

We wish to obtain an initial $A$-algebra
\[A \xto{i} S^{-1}A\]
such that the $i(s)$ are invertible; a pseudo-
\[ \{\frac{a}{s} \suchthat a \in A, s \in S \}.\]
Indeed, in the case that $A$ is a domain, the preceding definition is entirely satisfactory.

\begin{construction}[localization of a ring at a multiplicative subset]
  \mbox{}
  
  Step 1.
Let
\[I = \{ a \in A \suchthat \exists s \in S \enspace sa = 0 \}.\]

Note that if $a, b \in I$, then
\[ \exists s \in S \enspace sa = 0, \quad \exists t \in S \enspace tb = 0.\]
Then $(st)(a \pm b) = 0$.

Upshot: $I$ is an ideal.
We let
$A \xra{\pi} A/I =: A'$.

Step 2.
$\pi(S) \subseteq A'$
consists of non-zerodivisors.

Indeed, suppose that $s \in S$. If $w \in A'$ is such that $w \pi(s) = 0$, then from surjectivity of $\pi$ it follows that
\begin{align*}
  \exists a \in A \quad & w = \pi(a) \\
                        & \pi(s) \pi(a) = 0 \\
                        & \implies \pi(sa) = 0 \\
                        & \implies  sa \in I \\
                        & \implies \exists t \in S tsa = 0 \\
                        & \implies (ts)a = 0.
\end{align*}
But because $ts \in S$, it follows that $a \in I$ and so $\pi(a) = 0$. This means that $\pi(s)$ is not a zero divisor.

Define $S' := \pi(S)$. We want
\[ (S')^{-1} A' = \{ \frac{a'}{s'} \suchthat a' \in A', s' \in S' \}.\]
Consider pairs
\[\{(a', s') \suchthat a' \in A', s' \in S'\}\]
and define their relation $\sim$ thus:
\[(a'_1, s'_1) \sim (a'_2, s'_2) \iff a'_1 s'_2 = a'_2 s'_1.\]
To prove that this is an equivalence relation, we show transitivity (exercise; use that the $S'$ are non-zerodivisors).

Step 3.
\[(S')^{-1}A' := \{(a', s') \suchthat a' \in A', s' \in S' \} / {\sim}\]
is a well defined set.
The ring operations are given by thinking about this as
\[ \{ \frac{a'}{s'} \suchthat a' \in A', s' \in S'\}.\]
This turns the set into an associative commutative ring with unity.
\end{construction}

\begin{lemma}
  Let $a, b \in A$, $s, t \in S$. Then
  \[ \frac{a}{s} = \frac{b}{t} \quad \iff \quad \exists u \in S \enspace u(at-bs)=0.\]
\end{lemma}
\begin{proof}
  ``$\impliedby$'':
  \begin{align*}
    u(at - bs) = 0 & \implies uta = usb \\
                   & \implies \frac{uta}{1} = \frac{usb}{1} \\
                   & \implies  \frac{a}{s} = \frac{uta}{stu} = \frac{usb}{stu} = \frac{b}{t}.
  \end{align*}
  ``$\implies$'': Observe that the map $\phi: A /{I} = A' \to (S')^{-1}A^{-1}$ is injective.
  Note that
  \[ \forall \bar a, \bar b \in A' \enspace \frac{\bar a}{1} = \frac{\bar b}{1} \iff \bar a = \bar b \iff 1 \cdot \bar a = 1 \cdot \bar b \]
  % TODO
\end{proof}

\begin{corollary}
  A model of $S^{-1}A$ is given as the quotient
  \[ \{ (a, s) \suchthat a \in A, s \in S \} /{\sim}\]
  under
  \[ (a, s) \sim (b, t) \quad \iff \quad \exists s \in S \enspace u(at - bs) = 0.\]
\end{corollary}

\begin{example}
  For $f \in A$, let $S = \{ 1, f, f^2, \ldots \}$. We denote $S^{-1}A$ by $A_f$.
\end{example}

\begin{example}
  $\mathfrak p \in \spec(A)$,
  $S \coloneqq A \setminus \mathfrak p$.
  Then we denote $S^{-1}A$ by $A_{\mathfrak p}$.
\end{example}

\begin{note}
  $S^{-1}A$ does not determine $S$, for instance
  \[ \complex [x]_x = \complex [x]_{x^{2019}} \]
\end{note}

\begin{prop}[universal property of localization]
  \label{loc-univ}
  Let $f: A \to B$ be such that $f(S)$ consists of invertible elements.
\begin{equation*}
  \begin{tikzcd}
    A \arrow{rr}{f} \arrow[swap]{dr}{i}
    & & B \\
    & S^{-1}A \arrow{ru}[swap]{S^{-1}f}
    & \\
  \end{tikzcd}
  \end{equation*}
  Then $\exists ! S^{-1} \colon S^{-1}A \to B$
  such that
  $f = (S^{-1}f) \circ i$.
\end{prop}
\begin{proof}
  We write $\tilde f \coloneqq S^{-1}f$. Suppose that $\tilde f$ exists. Then
  \[ \forall a \in A \enspace \tilde f (a / 1) = f(a) \]
  and thus
  \[ \forall s \in S \enspace \tilde f(\frac{a}{s} \cdot s) = f(\frac{a}{1}) = f(a) \]
  and the formula
  $\tilde{f}(a / s) = f(a) / f(s)$
  is recovered.
  One needs only to check that such a function is a homomorphism.
\end{proof}

\begin{corollary}
  Consider the category of $A$-algebras whose structure maps invert $S$. Then \cref{loc-univ} reads: $S^{-1}A$ is the initial object of this category.
\end{corollary}

\begin{lemma}
  \label{loc-1}
  $A_f \cong A[x] / {(fx-1)}$.
\end{lemma}

\begin{proof}
  Note that $f$ is invertible in $A[x] / {(fx-1)}$.
  Hence by universal property
  \[\tilde{\phi}(a / f) = \phi(a) / \phi(f) = \bar{ax}. \]
  Now take
  $\psi \colon A[x] \to A_f$ a homomorphism of $A$-algebras defined by $\psi(x) = 1 / f$.
  Then
  \[ \psi(fx - 1) = f \cdot 1 / f - 1 = 0 \]
  and so
  \[ \exists \tilde \psi \colon A[x] /{(xf-1)} \to A_f \]
  such that
  \[ \forall a \in A \enspace \forall n \in \naturals \enspace \tilde \psi (a x^n) = a / f^n.\]
  The maps $\tilde \phi$ and $\tilde \psi$ are then mutual inverses.
\end{proof}

\begin{corollary}
  $A_f$ is a finitely generated \(A\)-algebra.
\end{corollary}

\begin{lemma}
  For $A$ an algebra and $S$ a multiplicative subset:
  \begin{enumerate}
  \item $S^{-1}A = 0 \iff 0 \in S$
  \item $A \xra{i} S^{-1}A$ is injective if and only if all elements of $S$ are non-zerodivisors in $A$
  \item $A \to S^{-1}A$ is an isomorphism if and only if all elements of $S$ are invertible in $A$
  \end{enumerate}
\end{lemma}
\begin{proof}
  \begin{enumerate}
  \item By \cref{loc-1}:
    \[ \frac{1}{1} = \frac{0}{1} \iff \exists u \in S \enspace u(1 \cdot 1 - 0 \cdot 1 ) = 0 \iff \exists u \in S u = 0.\]
  \item $i$ is injective if and only if
    \[ (\{a \in A \suchthat \exists s \in S \enspace sa = 0 \}).\]
    This happens if and only if $s$ is a non-zerodivisor.
  \item If $A \to S^{-1}A$ is iso, then $\forall s \in S \enspace s/1 \text{ is invertible}$.
    Hence, so is $s$.

    Conversely, if all elements of $S$ are invertible in $A$, then they are non-zerodivisors. Hence $A \to S^{-1}A$ is injective by the previous point.
    Moreover,
    \[i (a s^{-1}) = \frac{a s^{-1}}{a} = \frac{a}{s},\]
    so $i$ is ``onto''.
  \end{enumerate}
\end{proof}

Intuitively, we think of $A_f$ as the ring of functions on $X_f$, where $X = \spec(A)$.

We would expect $\spec(A_f)$ to be the same as $X_f$. Indeed, the following holds.

\begin{prop}
  \label{loc-2}
  Consider the maps
  \begin{gather*}
    i \colon A \to S^{-1}A \\
    i^* \colon \spec(S^{-1}A) \to \spec(A).
  \end{gather*}
  Then $i^*$ is injective and
  \[ \im i^* = \{ \mathfrak p \suchthat \mathfrak p \cap S = \emptyset \}. \]
\end{prop}

Before presenting the proof, for which we will require some additional facts, we note the following corollaries.

\begin{corollary}
  If $S = \{1, f, f^2, \ldots \}$, then
  \[ \im i^* = \{ \mathfrak p \suchthat \mathfrak p \not\ni f\} = (\spec(A))_f. \]
\end{corollary}

\begin{corollary}
  For $\mathfrak q \in \spec(A)$, $S = A \setminus q$,
  \[ \im i^* = \{ \mathfrak p \suchthat \mathfrak p \subseteq \mathfrak q \} = \spec(A_p). \]
\end{corollary}

\begin{lemma}
  \label{loc-3}
  Let $I \subseteq S^{-1}A$ be an ideal. If we let
  \[ J = \{ a \in A \suchthat \frac{a}{1} \in I \}, \]
  then
  \[ I = \{ \frac{j}{s} \suchthat j \in J, s \in S\}. \]
\end{lemma}
\begin{proof}
  ``$\supseteq$'':
  \[ \forall j \in J \enspace \frac{j}{i} \in I \implies \frac{j}{s} = \frac{j}{1} \cdot \frac{1}{s} \in I \]
  ``$\subseteq$'':
  \[ \frac{a}{s} \in I \implies \frac{a}{1} = \frac{a}{s} \cdot \frac{s}{i} \in I \implies a \in J \implies \frac{a}{s} \in \{\frac{j}{s} \suchthat j \in J, s \in S \}\]
\end{proof}

\begin{proof}[\cref{loc-2}]
  Let $\mathfrak q \in \spec(S^{-1}A)$. We let
  \[ \mathfrak p = i^*(\mathfrak q) = i^{-1}(\mathfrak q) = \{a \in A \suchthat f \in \mathfrak q \}.\]
  By \cref{loc-3} we can recover $\mathfrak q$ from $\mathfrak p$ alone, and so $i^*$ is injective.

  Let
  $\mathfrak p = i^* (\mathfrak q)$.
  Suppose
  $\mathfrak p \cap S \neq \emptyset$.
  This means that
  \[\frac{s}{1} \in \mathfrak q \text{ and } \frac{s}{t}  \text{ is invertible } \implies \mathfrak q = (1).\]

  Suppose $\mathfrak p \in \spec(A)$, $\mathfrak p \cap S = \emptyset$.
  We wish to find $\mathfrak q$. We guess that
  \[ q \coloneqq \{ \frac{p}{s} \suchthat p \in \mathfrak p, s \in S \} \subseteq S^{-1}A\]
  does the job.
  This $\mathfrak q$ is an ideal, we want to see that it is prime.

  Suppose that
  \[ \frac{a}{s} \cdot \frac{b}{t} \in \mathfrak q .\]
  Then
  $ab / 1 \in \mathfrak q$
  and so
  %TODO
\end{proof}

\begin{corollary}
  For $f \in A$, the following are equivalent:
  \begin{enumerate}
  \item $f$ is nilpotent (that is, $\exists n > 0 \enspace f^n = 0$),
  \item $f \in \bigcap \{\mathfrak p \suchthat p \in \spec(A) \}$,
  \item $V(f) = \spec(A)$.
  \end{enumerate}
\end{corollary}

\begin{proof}
  The equivalence between 2 and 3 is checked formally.

  ``$1 \implies 2$'': $\forall \mathfrak p \enspace f^n = 0 \in \mathfrak p \implies f \in \mathfrak p$.

  `` $\neg 1 \implies \neg 2$":
  let
  $0 \notin \{ 1, f, f^2, \dotsc \}$
  so $A_f \neq 0$ by \cref{loc-3}.
  Hence, $\spec(A_f) \neq \emptyset$, so by \cref{loc-2}
  \[ \spec(A_f) = \{ \mathfrak p \in \spec(A) \suchthat f \notin \mathfrak p \} \neq \emptyset .\]
\end{proof}

\begin{df}
  The nilradical of a ring $A$ is defined as
  \[ \text{nil}(A) = \bigcap \{ \mathfrak p \suchthat p \in \spec(A) \} = \{f \in A \suchthat \exists n > 0 \enspace f^n = 0 \}.\]
\end{df}

%%% Local Variables:
%%% mode: latex
%%% TeX-master: "../commalg"
%%% End:
