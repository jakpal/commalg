\section{Tensor product}
Fix $A$-modules $M, N$. We will construct their tensor product $M \tensor N$ - another $A$-module whose definition demands that it in some way classify ``$A$-multiplications'' with domain $M \times N$.

\begin{df}
  \label{def-bilinear}
  Let $P$ be an $A$-module. A function (not a homomorphism!)
  \[ f \colon M \times N \to P\]
  is called $A$-bilinear if:
  \begin{enumerate}
  \item $\forall a \in A, m \in M, n \in N \enspace a f(m, n) = f(am, n) = f(m, an)$,
  \item $f(m_1 + m_2, n) = f(m_1, n) + f(m_2, n)$,
  \item $f(m, n_1 + n_2) = f(m, n_1) + f(m, n_2)$.
  \end{enumerate}
  The set of such functions will be denoted by
  \[ \operatorname{Bilin}_{M \times N} (P).\]
\end{df}

\begin{prop}
  \label{tensor-pullback}
  For a homomorphism of $A$-modules $\phi \colon P \to R$ one gets a map
  \[ \phi^* \colon \bilin_{M \times N}(P) \to \bilin_{M \times N}(R) \quad \text{via} \quad f \mapsto \phi \circ f.\]
\end{prop}

\begin{theorem}
  \label{tensor-exists}
  There exists a unique (up to isomorphism) pair
  \[ T \in \module{A}, \quad g \in \bilin_{M \times N}(T)\]
  such that
  \[ \quad \forall P \in \module{A} \enspace \forall f \in \bilin_{M \times N}(P) \enspace \existsunique f' \colon T \to P \enspace f = f' \circ g.\]
\end{theorem}

\begin{note}
  One can rephrase \cref{tensor-exists} as the existence of a natural isomorphism
  \[ \hom(T, -) \simeq \bilin_{M \times N}(-).\]
\end{note}

\begin{df}
  The $A$-module $T$ of \cref{tensor-exists} is denoted by
  \[ M \tensor_A N \]
  and called the tensor product of $M$ and $N$.
  The map $g$ is written as
  \[ m \tensor n \coloneqq g(m, n).\]
\end{df}

\begin{proof}(\cref{tensor-exists})
  Let
  \[ T_0 = A^{M \times N}.\]
  Then
  \[ \forall P \enspace \hom(T_0, P) \simeq \set(M \times N, P) \quad \text{via} \quad  f \mapsto \sum a_i f(e_i)\]
  as the free module functor is adjoint to the forgetful functor to $\set$.
  Denote the generator of $T_0$ corresponding to $(m, n)$ by $e(m, n)$. Now define
  \[ T = T_0 / {\sim}, \]
  where $\sim$ describes the relations given in \cref{def-bilinear}.
  We now claim that $T$ together with the quotient of $e$ satisfies the universal property.

  Uniqueness up to isomorphism does follow from this very universal property.
\end{proof}

\begin{question}
  Do all representable functors have left adjoints?
\end{question}
\begin{answer}
  Yes, provided the category considered has all coproducts, as then for functors into $\set$ the two conditions are in fact equivalent; the same holds in the case of corepresentability and having a right adjoint.

  Concretely, the left adjoint to a functor represented by $X$ is given by
  \[ Y \mapsto \disjointunion_Y X.\]
  Note that this does not agree with our definition of a tensor product. Indeed, while the representable functors considered in this question are Set-valued, the one right adjoint to the tensor product is an endofunctor of the category of $A$-modules.
\end{answer}



%%% Local Variables:
%%% mode: latex
%%% TeX-master: "../commalg"
%%% End:
