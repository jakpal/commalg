\section{The Spec() functor}
We aim to mirror the previous considerations of \cref{mot-circle} in the case of an arbitrary ring.

\begin{df}
Let $A$ be a ring. The \textit{spectrum} of $A$ is its set of prime ideals:
\[\spec(A) = \{ \mathfrak p \text{ a prime ideal in } A \}.\]
\end{df}

\begin{example}
  $\spec(\complex [x]) =  \{0\} \cup \{ (xf-a) \suchthat a \in \complex \}$.
\end{example}

For a ring homomorphism $f \colon A \to B$, \cref{primes are preserved under preimage} asserts that there is a map
\[f^* \colon \spec(B) \to \spec(A)\]
given by 
\[B \supseteq \mathfrak q \mapsto f^{-1}(\mathfrak q) \subseteq A.\]

We will now move towards upgrading spectra of rings (which up to this point we considered as mere sets) to topological spaces in such a way that makes the maps $f^*$ continuous. In this, we follow the previous motivation.

\begin{df}
  \label{def vanishing locus}
For $E \subseteq A$ an arbitrary subset, we define the \textit{vanishing locus of $E$} as the set
\[V(E) = \{ \mathfrak p \in \spec(A) \suchthat E \subseteq \mathfrak p \} .\]
\end{df}

\begin{prop}
  \label{zariski-prop}
  The sets $V(E)$, $E \subseteq A$, are the closed subsets of a topology.
\end{prop}

\begin{proof}
  One sees immediately that
  \[ V(1) = \emptyset, \quad V(0) = \spec(A).\]
  Elementary set-theoretic considerations reveal that
  \[\cap_{i \in I} V(E_i) = V(\cup_{i \in I} E_i).\]
  It now suffices to show that for
  \(E_1 \cdot E_2 \coloneqq \{ e_1 \cdot e_2 \suchthat e_1 \in E_1, e_2 \in E_2 \},\)
  one has
  \[ V(E_1) \cup V(E_2) = V(E_1 \cdot E_2).\]
  The inclusion $V(E_1) \cup V(E_2) \subseteq V(E_1 \cdot E_2)$ is easy to verify.
  Suppose without loss of generality that $\mathfrak p \in V(E_1)$, that is
  $\mathfrak p \supseteq E_1.$
  Then
  \[ E_1 \cdot E_2 \subseteq \mathfrak p \cdot E_2 \subseteq \mathfrak p,\]
  because $\mathfrak p$ is an ideal.
  In order to prove the other inclusion, we need to use that $\mathfrak p$ is prime.
  Suppose that
  \( \mathfrak p \not\supseteq E_1\) and \(\mathfrak p \not\supseteq   E_2.\)
  Then there exist $e_i \in E_i \setminus \mathfrak p$, $i=1,2$. For those,
  \( e_1 \cdot e_2 \in E_1 \cdot E_2 \setminus \mathfrak p\)
  holds because $\mathfrak p$ is prime.
\end{proof}

\begin{df}
  The topology on $\spec(A)$ defined by \cref{zariski-prop} is called the \textit{Zariski topology}.
\end{df}

\begin{question}
What is the Zariski topology for $A = \complex [x]$?
\end{question}

\begin{answer}
We note that
\begin{align*}
  V(f_1, \ldots, f_r) & = \{ \mathfrak p \text{ prime in } A \suchthat f_1, \dotsc, f_r \in \mathfrak p\} \\
                      & = \{(x-a) \text{ maximal in } A \suchthat f_1, \dotsc, f_r \in (x-a)\} \\
                      & = \{ (x-a) \suchthat f_1(a) = \dotsc = f_r(a) = 0 \}.
\end{align*}

If any $f_i$ is nonzero, then $0 \notin V(f_1, \dotsc, f_r)$.

Hence,
\[ V(f_1, \ldots, f_r) = \text{ set of common roots of }f_1, \ldots, f_r \]
and one sees that $V(f_1, \ldots, f_r)$ is finite.

In fact, all finite subsets of the set of maximal ideals are of the form $V(f_1, \ldots, f_r)$. Note that the only closed set containing the prime ideal $0$ is $\spec(A)$ itself; in other words, $0$ lies in every nonempty open set. This yields the cofinite topology augmented by $\{0\}$.
\end{answer}

\begin{note}
  Because the closure of $\{0\} \subseteq \spec(A)$ is $\spec(A)$ itself, the resulting space is not T1, let alone Hausdorff. It is, however, T0; this is simply because given two distinct prime ideals, one of them is not contained inside another, say \(\mathfrak{p}_1 \not\subseteq \mathfrak{p}_2.\) Then, \(\mathfrak{p}_1 \in V(\mathfrak{p}_1)\) and \(\mathfrak{p}_2 \notin V(\mathfrak{p}_2).\)
\end{note}

\begin{prop}
  Let $f \colon A \to B$ be a ring homomorphism. Then the induced map
  \[f^* \colon \spec(B) \to \spec(A)
    \quad \text{with} \quad
    \mathfrak q \mapsto f^{-1}(\mathfrak q)\]
  is continuous.
\end{prop}
\begin{proof}
It suffices to see that the preimage of every closed set is closed. Indeed, we claim that
\[ (f^*)^{-1}(V(E)) = V(f(E)).\]
We unravel the definitions:
\begin{align*}
  (f^*)^{-1}(V(E)) & = \{ \mathfrak p \subseteq B \suchthat f^*(\mathfrak p) \in V(E)\} \\
                   & = \{ \mathfrak p \subseteq B \suchthat f^{-1}(\mathfrak p) \in V(E)\} \\
                   & = \{ \mathfrak p \subseteq B \suchthat E \subseteq f^{-1}(\mathfrak p) \} \\
                   & = \{ \mathfrak p \subseteq B \suchthat f(E) \subseteq \mathfrak p\} \\
                   & = V(f(E)).
\end{align*}
Note that the considerations here are purely set-theoretical. The condition that $f$ be a ring homomorphism is only relevant for $f^*$ to be well defined on spectra.
\end{proof}

\begin{prop}
  \label{spec basis}
  Let \(R\) be a ring. The topological space \(\spec(R)\) has a basis of open sets given by
  \[\{D(f) \suchthat f \in R\}, \quad \text{where} \quad D(f) = \spec(R) \setminus V(f).\]
\end{prop}
\begin{proof}
  Any open set is of the form
  \[\spec(R) \setminus V(E) = \setsum_{f \in E} D(f).\]
\end{proof}

\begin{prop}
  \label{spec quasi-compact}
  Let \(R\) be a ring. The topological set \(\spec(R)\) is \emph{quasi-compact} in the sense that any open cover has a finite subcover.
\end{prop}
\begin{proof}
  It is enough to prove the claim in the case of an open cover consisting of some sets \(D(f)\) of \cref{spec basis}. But then, if a family \(\{D(f)\}_{f \in B}\) covers \(\spec(R),\) it means that \(1 \in R\) is generated by the set \(B.\) This means that it is already generated by some finite subset \(A \subseteq B,\) in which case \(\{D(f)\}_{f \in A}\) is a finite subcovering.
\end{proof}


%%% Local Variables:
%%% mode: latex
%%% TeX-master: "../commalg"
%%% End:
