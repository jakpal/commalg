\section{Motivations}

\begin{lemma}
  \label{primes are preserved under preimage}
  Let $f:A \to B$ be a ring homomorphism. If $\mathfrak p \subset B$ is prime, then $f^{-1}(\mathfrak p) \subset A$ is prime.
\end{lemma}

\begin{proof}
  Consider the following diagram.
  \begin{equation*}
  \begin{tikzcd}
    A \arrow[r] \arrow[d]
    & B \arrow[d] \\
    A /{ f^{-1}(\mathfrak p) } \arrow[r]
    & B /{\mathfrak p}
  \end{tikzcd}
\end{equation*}
The preimage of an ideal is an ideal; moreover, the lower vertical map is a injective. Hence,
  $A /{f^{-1}(\mathfrak p)} \subseteq B /{\mathfrak p}$
  is a subring. If then $\mathfrak p$ is prime, $B /{\mathfrak p}$ is a domain; its every subring is then a domain as well. Thus,
$A / {f^{-1}(\mathfrak p)}$
is a domain, and so
$f^{-1}(\mathfrak p) \subseteq A$
is prime.
\end{proof}

\begin{example}
The inclusion $\integer \hookrightarrow \rational$ has $f^{-1}(0) = 0$ not maximal, even though $0 \in \rational$ is maximal.
Ergo, \cref{primes are preserved under preimage} does not hold with "prime" replaced by "maximal".
\end{example}

\begin{example}
  \label{mot-circle}
  Consider
  \[ S^1 = \{ z \in \complex | |z|=1 \}, \quad A = C(S^1, \real) = \{ \text{ continuous real functions on the circle } \}.\]
  Then there is a bijection between $S^1$ and maximal ideals in $A$, given by every maximal ideal being of the form
  \[\mathfrak m_x = \{ f \in A \suchthat f(x) = 0 \} .\]

  Moreover, one can recover the topology of $S^1$ from $A = C(S^1, \real)$.

  Indeed, for $f \in A$, consider the vanishing set
  \[ V(f) = \{ \mathfrak m \text{ maximal ideal in $A$ } | f \in m \}.\]

  The topology is then generated by closed subsets with subbase
  \[\{V(f) \suchthat f \in A\}.\]
\end{example}

%%% Local Variables:
%%% mode: latex
%%% TeX-master: "../commalg"
%%% End:
