\section{Noether Normalization}

\begin{theorem}(Noether normalization; cf. \cite[\S 13]{eisenbud1995commutative})
  \label{Noether normalization}
  Let \(A\) be a finitely generated \(k\)-algebra. Assume that \(A\) is a domain.
  Consider
  \[\mathfrak p_1 \subseteq \dotsc \subseteq \mathfrak p_m\]
  a chain in \(A.\)
  Then, there exists a finite extension
  \[k[x_1, \dotsc, x_d] \subseteq A, \quad \dim(A) = d, \quad \dim(A/{\mathfrak p_i}) \eqqcolon d_i,\]
  such that
  \[\forall i \enspace \mathfrak p_i \cap k[x_1, \dotsc, x_d] = (x_{d_i+1}, x_{d_i+2}, \dotsc, x_d)\]
  and with
  \(k[x_1, \dotsc, x_{d_i}] \subseteq A/{\mathfrak p_i}\)
  a finite extension.
\end{theorem}

Intuitively, this lets us think of ideals in a chain as if they were exactly the planes, curves and points, with their respective interplay preserved by a ``coordinate map'', leading from the variety into an affine space of equal dimension. That the algebraic extension is finite corresponds to fibers of this morphism being finite.

\begin{note}
  In the case that the prime ideal chain consists only of the zero ideal, \cref{Noether normalization} states that \(A\) is a finite extension of a polynomial ring.
\end{note}


\begin{theorem}[weak Nullstellensatz]
  \label{weak Nullstellensatz}
  If \(A\) is a finitely generated \(k\)-algebra, \(\mathfrak m \subseteq A\) a maximal ideal, then
  \(k \into A/{\mathfrak m}\)
  is finite.
  In particular, if \(k = \bar k,\) then \(k \cong A/{\mathfrak m}.\)
\end{theorem}
\begin{proof}
  Since \(A/{\mathfrak m}\) is a finitely generated \(k\)-algebra, we have
  \(\dim(A/{\mathfrak m}) = 0.\)
  Apply \cref{Noether normalization} to get a finite extension
  \(k[x_1, \dotsc, x_d] \subseteq A/{\mathfrak m}\)
  wth \(\dim(A/{\mathfrak m}) = 0.\)
  Then
  \(k \subseteq A/{\mathfrak m}\)
  is finite.
  Suppose that \(k = \bar k.\) Should \(k \into A/{\mathfrak m}\) not be an equality, then there exists
  \(\alpha \in A/{\mathfrak m} \setminus k\)
  together with its minimal polynomial over \(k:\)
  \[\alpha^r + k_{r-1} \alpha^{r-1} + \dotsc + k_0 = 0, \quad k_i \in k.\]
  By Bezout, there exists a \(\beta \in k\) such that
  \[\beta^r + k_{r-1} \alpha^{r-1} + \dotsb + k_0 = 0.\]
  Then
  \[ \alpha^r + k_{r-1} \alpha^{r-1} + \dotsb + k_0 = (\alpha - \beta)(\dotso) = 0.\]
  Either \(\alpha = \beta\) or \((\dotso)\) is satisfied.
  % TODO ???
\end{proof}


\begin{corollary}
  If \(k\) is algebraically closed, then
  \(\spec_{max}(k[x_1, \dotsc, x_d]) = k^d.\)
\end{corollary}
\begin{proof}
  Take
  \(\mathfrak m \subseteq k[x_1, \dotsc, x_d]\)
  a maximal ideal. By \cref{weak Nullstellensatz}, the map
  \(k \to k[x_1, \dotsc, x_d]/{\mathfrak m}\)
  is an isomorphism.
  If one considers the quotient map
  \(k[x_1, \dotsc, x_d] \onto k[x_1, \dotsc, x_d]/{\mathfrak m},\)
  this means that
  \(\forall x_i - \alpha_i \in \mathfrak m\)
  and so
  \((x_1 - \alpha_1, \dotsc, x_d - \alpha_d) \subseteq \mathfrak m.\)
  Since both ideals are maximal, there must be an equality.
\end{proof}


We will now move towards a proof of the Noether normalization theorem. First, we give some intuitions.

Even if \(A = k[x_1, \dotsc, x_d]\) and each \(\mathfrak p_i\) is generated by linear forms, there is a coordinate change involved. We will claim that solving this problem guides us towards the general solution.

\begin{proof}(Noether normalization)
  \emph{Case 1.} Let \(A = k[y_1, \dotsc, y_d]\) a polynomial ring. We claim that there exist \(x_1, \dotsc, x_d \in A\) such that
  \begin{enumerate}
  \item \(k[x_1, \dotsc, x_d] \subseteq A\) is finite,
  \item \((x_{d_i+2}, x_{d_i+2}, \dotsc, x_d) \subseteq \mathfrak p\) and \(d_i = \dim (A/{\mathfrak p_i}).\)
  \end{enumerate}
  Indeed: let \(x_1' \coloneqq y_1, \dotsc, x_d' \coloneqq y_d.\)
  We find \(x_1, \dotsc, x_d\) by downward induction.
  Suppose that we found
  \(x_d, \dotsc, x_{e+1}, x_e', \dotsc, x_1'\) so that:
  \begin{enumerate}
  \item \(k[x_1', \dotsc, x_e', x_{e+1}, \dotsc, x_d] \subseteq A\) is finite,
  \item \(\mathfrak p_i \supseteq (x_n, \dotsc, x_d),\) \(h = \max(d_i+1, e + 1).\)
  \end{enumerate}
  For the base of induction, point 1. is true and point 2. - vacuous.
  \emph{Induction step}: we want to find \(x_e.\)
  Let \(i\) be the smallest index such that \(x_e\) lies in \(\mathfrak p_i,\) that is, the smallest \(i\) such that
  \(d_i \leq e - 1.\)
  Put
  \(S_e \coloneqq k[x_1', \dotsc, x_e', x_{e+1}, \dotsc, x_d].\)
  We have
  \[d_i = \dim(A/{\mathfrak p_i}) = \dim(S_e/{S_e \cap \mathfrak p_i}).\]
  Point 2. implies that
  \(x_{e+1}, \dotsc, x_d \in \mathfrak p_i,\)
  and so in fact
  \(d_i = \dim(k[x_1', \dotsc, x_e']/{\mathfrak p_i \cap k[x_1', \dotsc, x_e']}) \leq e - 1.\)
  Hence,
  \(\mathfrak p_i \cap k[x_1', \dotsc, x_e'] \neq 0.\)
  Choose
  \(x_e \in \mathfrak p_i \cap k[x_1', \dotsc, x_e'].\)
  Now, perform Nagata's coordinate change to get a finite extension
  \[k[x_1'', \dotsc, x_{e-1}'', x_e] \subseteq k[x_1', \dotsc, x_e'].\]
  We replace \(x_1, \dotsc, x_e'\) by \(x_1'', \dotsc, x_{e-1}'', x_e.\)
  For those, point 1. holds because the extension is finite; point 2, because \(x_e \in \mathfrak p_i.\)
  We have finished the induction step, and so the first claim of \emph{Case 1} is proved.

  Now, what follows is that:
  \begin{enumerate}
    \item \(\dim(k[x_1, \dotsc, x_d]) = \dim(A) = d,\)
      so there are no relations among \(x_i\),
    \item One has
      \begin{align*}
        d_i = \dim(A/{\mathfrak p_i}) & = \dim(k[x_1, \dotsc, x_d]/{\mathfrak p_i \cap k[x_1, \dotsc, x_d]})\\
                                      & = \dim(k[x_1, \dotsc, x_{d_i}]/{\mathfrak p_i \cap k[x_1, \dotsc, x_{d_i}]}),
      \end{align*}
      so \(\mathfrak p_i \cap k[x_1, \dotsc, x_{d_i}] = 0.\)
      Hence, \(\mathfrak p_i \cap k[x_1, \dotsc, x_d] = (x_{d_i+1}, x_{d_i+2}, \dotsc, x_d)\) and the proof of \emph{Case 1} is finished.
  \end{enumerate}

  \emph{Case 2} (of the general ring \(A\)).
  Write
  \(A = k[y_1, \dotsc, y_r]/{I}\)
  as \(A\) is finitely generated.
  \(I\) is prime, because \(A\) is a domain.
  W now lift
  \(\mathfrak p_1 \subseteq \dotsc, \mathfrak p_m\)
  to \(k[y_1, \dotsc, y_r],\) adding also \(I\):
  \(I \subseteq \mathfrak q_1 \subseteq \dotso \subseteq \mathfrak q_m.\)
  We now get
  \(k[x_1, \dotsc, x_r] \subseteq k[y_1, \dotsc, y_r]\) finite, and so
  \[I \cap k[x_1, \dotsc, x_r] = (x_{d+1}, x_{d_2}, \dotsc, x_r),\]
  \(\mathfrak q_i \cap k[x_1, \dotsc, x_r] = (x_{d_i+1}, x_{d_i+2}, \dotsc, x_r),\)
  \(d = \dim(A) = \dim(k[y_1, \dotsc, y_r])/{I}.\)
  We get a diagram
  \begin{equation*}
    \begin{tikzcd}[column sep = huge]
      k[x_1, \dotsc, x_r] \arrow{r}{\text{finite}} \arrow{d}
      & k[y_1, \dotsc, y_r] \arrow{d} \\
      k[x_1, \dotsc, x_d] \arrow{r}{\text{finite}}
      & k[y_1, \dotsc, y_r]/{I} = A
    \end{tikzcd}
  \end{equation*}
  In this, the vertical maps are surjective.
  Then
  \(\mathfrak p_i \cap k[x_1, \dotsc, x_d]\) is the image of \(\mathfrak q_i \cap k[x_1, \dotsc, x_r]\)
  so
  \(\mathfrak p_i \cap k[x_1, \dotsc, x_d] = (x_{d_i+1}, x_{d}).\)
  % TODO może być błąd bo szybko starte
\end{proof}


\begin{note}
  Let \(A\) be a ring. Then
  \[\nil(A) = \bigcap_{\mathfrak p \text{ prime}} \mathfrak p = \{f \suchthat \exists n \enspace f^n = 0\}.\]
  Take \(I \subseteq A,\) \(\pi \colon A \onto A/{I}.\)
  Then
  \[\pi^{-1}(\nil(A/{I})) = \{f \in A \suchthat \exists n \enspace f^n \in I\} = \bigcap_{\mathfrak p \in V(I)} \mathfrak p.\]
\end{note}

\begin{df}
The \emph{radical} of \(I\) is
\[\rad{I} = \{f \in A \suchthat \exists n \enspace f^n \in I\} = \bigcap_{I \subseteq \mathfrak p} \mathfrak p.\]
\end{df}
In particular, \(\rad{I}\) can be recovered from \(V(I).\)

\begin{prop}
  For \(\mathfrak p\) prime, \(\rad{\mathfrak p} = \mathfrak p.\)
\end{prop}

\begin{theorem}[Nullstellensatz]
  \label{Nullstellensatz}
  If \(A\) is a finitely generated \(k\)-algebra, \(k = \bar k\), \(I \subseteq A,\) then
  \[\rad{I} = \bigcap \{\mathfrak m \suchthat \mathfrak m \text{ maximal in \(A\)}, I \subseteq \mathfrak{ m}\}.\]
\end{theorem}

\begin{proof}
  \emph{Case 1}, \(A = k[x_1, \dotsc, x_d]\) a polynomial ring.
  If \(I\) is prime, denote \(I \eqqcolon \mathfrak p.\) \(\rad{\mathfrak p} = \mathfrak p,\) so we want
  \(\mathfrak p = \bigcap_{\mathfrak m \supseteq \mathfrak p} \mathfrak m.\)
  Obviously, \(LHS \subseteq RHS.\)
  % TODO rewrite
  Take
  \(f \in \bigcap_{\mathfrak m \supseteq \mathfrak p} \mathfrak m.\) Then
  \[(k[x_1, \dotsc, x_d]/{\mathfrak p})_{\mathfrak p} \cong k[x_1, \dotsc, x_d, y]/{(\mathfrak p + (fy-1))} \neq 0.\]
  Take a maximal ideal in
  \(k[x_1, \dotsc, x_d, y]/{(\mathfrak p) + (fy-1)}.\)
  It is te image of some maximal ideal in
  \(k[x_1, \dotsc, x_d, y].\)
  By \cref{weak Nullstellensatz}, this maimal ideal is
  \((x_1 - \alpha_1, \dotsc, x_d - \alpha_d, y - \beta),\)
  where
  \(\alpha, \dotsc, \alpha_d, \beta \in k.\)
  Now,
  \[(x_1 - \alpha_1, \dotsc, x_d - \alpha_d, y - \beta) \cap k[x_1, \dotsc, x_d] = (x_1 - \alpha_1, \dotsc, x_d - \alpha_d),\]
  so \(\mathfrak p \subseteq (x_1 - \alpha_1, \dotsc, x_d - \alpha_d).\)
  Suppose that
  \(f \in (x_1 - \alpha_1, \dotsc, x_d - \alpha_d).\)
  Then
  \[f \in (x_1 - \alpha_1, \dotsc, x_d - \alpha_d, y-\beta),\]
  \[fy-1 \in (x_1 - \alpha_1, \dotsc, x_d - \alpha_d, y - \beta),\]
  so \(1 \in (x_1 - \alpha_1, \dotsc, x_d - \alpha_d, y - \beta).\)
  Hence,
  \(f \notin (x_1 - \alpha_1, \dotsc, x_d - \alpha_d) \eqqcolon \mathfrak m.\)
  Now,
  \(m \supseteq \mathfrak p, f \notin \mathfrak p, f \notin \bigcap_{\mathfrak m \supseteq \mathfrak p} \mathfrak m.\)
  Thus,
  \(\bigcap_{\mathfrak m \supseteq \mathfrak p} \mathfrak m = \mathfrak p.\)
  Case 1 is finished, thus:
  \[\rad{I} = \bigcap_{\mathfrak p \supseteq I} \mathfrak p = \bigcap_{\mathfrak p \supseteq I} \bigcap_{\mathfrak m \supseteq \mathfrak p} \mathfrak m = \bigcap_{\mathfrak m \supseteq I} \mathfrak m.\]
  The general case is
  \(A = k[x_1, \dotsc, x_r]/{I};\) one lifts everything to \(k[x_1, \dotsc, x_r].\)
\end{proof}

\begin{example}
  If \(A = k[x]_{(x)},\) then \(0 \in A\) is prime, so \(\rad{0} = 0,\) but \[0 \neq \bigcap_{m \subseteq A \text{ maximal}} \mathfrak m = x A.\]
\end{example}




%%% Local Variables:
%%% mode: latex
%%% TeX-master: "../commalg"
%%% End:
